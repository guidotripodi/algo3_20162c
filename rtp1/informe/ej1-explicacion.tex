Para solucionar este problema, se debe ver la combinaci\'on de viajes entre lados del puente (lado A, origen y lado B, destino) que nos permiten pasar a todos los integrantes
del grupo, del lado A al B en el menor tiempo posible. Siendo esta busqueda una tarea exponencial, buscamos la forma de poder disminuir los casos evaluados, acotandonos solo a los relevantes
para la consigna, aplicando de este modo la busqueda del m\'inimo a travez de backtracking.\\

Dadas las restricciones del problema, se decidieron aplicar las siguientes podas sobre el árbol de desiciones:
\begin{itemize}
	\item Las combinaciones evaluadas son, en el caso de enviar 2 personas, a {\bf duplas sin repeticiones}.
	\item Los viajes, tanto de paso de A a B como de B a A que generan un 'desbalance', son obviados; es decir en el caso que en alg\'un lado hay más canibales que arqueologos.
	
	\item Se tomam las decisiones que no repitan estados ya alcanzados dentro de una misma rama para {\bf no generar  ciclos}
	\item Las secuencias de viajes que tomen más tiempo que una previamente calculada se descartan.
\end{itemize}
