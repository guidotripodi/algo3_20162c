Para solucionar este problema, se debe ver la combinaci\'on de viajes entre lados del puente (lado A, origen y lado B, destino) que nos permiten pasar a todos los integrantes
del grupo, del lado A al B en el menor tiempo posible. Siendo esta busqueda una tarea exponencial, buscamos la forma de poder disminuir los casos evaluados, acotandonos solo a los relevantes
para la consigna, aplicando de este modo la busqueda del m\'inimo a travez de backtracking.\\

Dadas las restricciones del problema, se decidieron aplicar las siguientes podas sobre el árbol de desiciones:
\begin{itemize}
	\item Las combinaciones evaluadas son, en el caso de enviar 2 personas, a {\bf duplas sin repeticiones}.
	\item Los viajes, tanto de paso de A a B como de B a A que generan un 'desbalance', son obviados; es decir en el caso que en alg\'un lado hay más canibales que arqueologos.
	
	\item Se tomam las decisiones que no repitan estados ya alcanzados dentro de una misma rama para {\bf no generar  ciclos}
	\item Las secuencias de viajes que tomen más tiempo que una previamente calculada se descartan.
\end{itemize}

A continuación se detallan los puntos anteriores:


\subsubsection*{Combinaciones posibles}
En cada viaje a travez del puente, a priori se puede mandar cualquier persona, o combinación de 2 personas. Para enumerar estos casos u opciones, podemos plantear una matriz de $N\timesN$ para simbolizar en cada casilla cada combinación posible. Es inmediato ver que las casillas de un lado y otro de la diagonal de la matriz repiten casos y otras en la diagonal de la misma, que repite la persona. Como también existe la posibilidad de enviar de a 1 persona por vez, entonces la diagonal de la matriz representará estas elecciones, por otro lado, nos quedaremos con solo 1 mitad de la matriz para no tener en cuenta los casos repetidos. Bajo esta representación obtenemos todas las combinaciones validas de personas que pueden llegar a viajar en cada cruce.

Como las personas solo pueden encontrarse en 1 isla en todo momento, entonces de todas las opciones se consideran aquellas que posean a todos los integrantes en el lado de partida.

\subsubsection*{Desbalance}
Al no poder haber más caníbales que arqueologos en ningun lado en ningún momento, la Elección tomada deberá mantener este invariante al enviar para el otro lado las personas elegidas.

\subsubsection*{Ciclos}
En el caso en que se envíe la opcion $(i; j)$ al lado A, bajo las restricciones anteriores nada impide que en el paso siguiente se decida mandar de regreso al mismo par, generandonos un ciclo. Esta elección claramente carece de sentido ya que nos regresa a un estado previo y con más tiempo acumulado. Podemos generalizar el ejemplo anterior a decir que no tiene sentido, dentro de una rama de solución, repetir un estado previo. Para evitarlo se debe guardar los estados previos por los cuales se pasó y desestimar cualquier opción que nos retorne a ellos.

\subsubsection*{Tiempo}
De haber encontrado una solución, no tendrá sentido ,al estar buscando una nueva, seguir por una rama si se supera el tiempo de la misma, ya que de alcanzar una solución, la misma no será optima. Con lo cual, se desestima la busqueda por una rama si la misma supera en tiempo alguna solución ya encontrada.