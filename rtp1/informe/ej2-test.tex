\indent Luego de realizar la implementaci\'on de nuestro algoritmo, desarrollamos tests,
para corroborar que nuestro algoritmo es el indicado.\\

A continuaci\'on enunciaremos varios de nuestros tests:\\

\begin{center}
 \textbf{El valor de entrada $P$ es de la forma ${3^i}$ para un i$\gets$[0, N] }
\end{center}
 Este caso se cumple cuando se recibe un P el cual al realizar nuestro primer ciclo que chequea cual es la potencia igual o mayor, termina siendo igual y de esta forma solo se itera una \'unica vez el segundo y tercer ciclo.
 
\begin{center}
 \textbf{El valor de entrada $P$ es de la forma \[
\sum_{i=1}^{n}3^{i}=P 
\]}
\end{center}

Veremos m\'as adelante que este caso ser\'a el peor a resolver ya que se iterar\'a la totalidad completa de elementos de nuestros arrays.

\begin{center}
 \textbf{El valor de entrada $P$ es m\'ultiplo de 3}
\end{center}

 Este caso se cumple cuando se recibe un $P MOD 3 = 0$.

\begin{center}
 \textbf{El valor de entrada $P$ no es m\'ultiplo de 3}
\end{center}

 Este caso se cumple cuando se recibe un P el cual el mismo es de la forma $P MOD 3 = 1$ o $P MOD 3 = 2$.
 