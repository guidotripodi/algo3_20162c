El algoritmo por cada uno de los $L = \sum_{i = 0}^N C_{i}$ elementos construye una matriz de $K_{1} \times K_{2} \times...\times K{m}$, y las guarda para poder recuperar al final, los elementos involucrados en la solución.
El costo total será:

\[
L \ast \prod_{i=1}^{M}K_{i}
\]

Dentro del algoritmo, por cada objeto a optimizar, se copiará la matriz de la optimización hecha con los objetos del paso anterior con lo cual tendremos un costo por iteración de:

\[
\prod_{i=1}^{M}K_{i}
\]

Como el algoritmo simplemente recorre toda la matriz por cada objeto actualizando los valores con operaciones en $O(1)$ el costo total final para todos los objetos será de:

\[
L \ast \prod_{i=1}^{M}K_{i}
\]

Si acotamos las capacidades de las mochilas por la de aquella de mayor capacidad (capacidad K) se tiene:

\[
\sum_{i = 0}^N(C_{i}) \ast \prod_{i=1}^{M} K_{i} \leq
L \ast \prod_{i=1}^{M}K_{i} = L \ast \prod_{i=1}^{M}K = L \ast  K^{M}
\]  

Dado que $K^{M}$ $<$ $(\sum K_{i})^M$ podemos concluir que el algoritmo respeta la cota requerida:


\[
\sum_{i = 0}^N(C_{i}) \ast  K^{M} \subseteq O( \sum C_{i} \ast (\sum K_{i})^M)
\]