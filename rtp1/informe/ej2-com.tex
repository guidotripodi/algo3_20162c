
Nuestro algoritmo como mencionamos anteriormente presenta dos ciclos predominantes de los cuales el segundo ciclo es el que realiza la busqueda de las pesas.\\

El primero ciclo consta en armar un arreglo de sumas de potencias de tres $sumasParciales$, donde la $i-esima$ posición es la suma desde $3^0$ hasta $3^i$ y la $n-esima$ será la suma desde $3^0$ hasta $3^n$  tal que es igual a $P$ (el valor de nuestra llave) o en su defecto el inmediato mayor. 

Podemos notar que para representar un número en base tres necesitamos $\ceil[\big]{log_3(P)}$ divisiones con lo que obtendremos un total de $\ceil[\big]{log_3(P)}$  potencias de tres. 
En nuestro algoritmo, puede ser necesario tomar la potencia inmediatamente mayor a $P$ por lo cual tendriamos $\ceil[\big]{log_3(P)}$ + 1 potencias de tres en el peor caso, aunque en terminos de complejidad sigue siendo orden logaritmico. 

Queremos ver entonces que  $\ceil[\big]{log_3(P)}$ es una buena cota para la cantidad de iteraciones que realizamos para obtener el arreglo $sumasParciales$ y en consecuencia, para realizar el ciclo que encuentra las pesas adecuadas. Es decir, que seguro será suficiente sumar a lo sumo $\ceil[\big]{log_3(P)}$ potencias de tres para obtener una suma que sea mayor que $P$. Por lo tanto:

\begin{equation}
P \leq \sum_{i=0}^{\ceil[\big]{log_3(P)}}(3^i)
\end{equation}

Tomando el último elemento de la suma tenemos que:

\begin{equation}
3^{\ceil[\big]{log_3(P)}} \geq 3^{log_3(P)} = P
\end{equation}

Por lo tanto se puede ver que el último termino de la sumatoria ya es más grande que P. Por lo tanto la suma total lo será. En consecuencia, a lo sumo será necesario sumar $\ceil[\big]{log_3(P)}$ potencias de tres.

Además como se sabe, en terminos de complejidad, los logaritmos en cualquier base son iguales al logaritmo en base diez, dado que la única diferencia es una constante multiplicativa. Por lo tanto 


\begin{equation}
O(\ceil[\big]{log_3(P)}) \subseteq O(\ceil[\big]{log(P)})
\end{equation}

Dado que el ciclo donde se obtienen las pesas se basa en recorrer todas las posiciones del arreglo $sumasParciales$ y además las operaciones realizadas son en orden constante, el orden de complejidad total será el mencionado.
Además, esta clase de complejidad esta incluida en la clase de complejidad $\sqrt{P}$. Por lo tanto se cumple con la complejidad pedida en el enunciado del problema.
