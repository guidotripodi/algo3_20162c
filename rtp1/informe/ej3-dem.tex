El problema de la mochila, en cualquier dimensi\'on, se basa en maximizar un valor total teniendo en cuenta una o varias capacidades de las cuales no podemos excedernos. Como queden completadas las mochilas involucradas en cuanto a peso se refiere no dice nada, a priori, del valor total.\\

La idea de aplicar una tecnica de programación dinámica en un problema de optimización es que podemos maximizar este valor total teniendo en cuenta lo mejor que pudimos hacer para subproblemas m\'as pequeños.\\

En el caso de una dimensi\'on, estos subproblemas serán más facil de ver, y por lo tanto abordaremos esta demostración desde la perspectiva de una mochila en primera instancia.\\

Nuestro subproblema en este caso es: que es lo mejor que se pudo lograr con $i$ objetos y capacidad $k$ con 0 $\leq k \leq K$, siendo $K$ la capacidadad total de la mochila.
Puede describirse de la siguiente manera:

\begin{center}
$S(i, k)$ = máximo beneficio obtenido con $i$ objetos para una capacidad disponible $k$
\end{center}

A cada subproblema asociamos un objeto con el cual haremos un análisis exhaustivo de las posibilidades, las cuales serán, sea un objeto $e_i$, si utilizo o no utilizo el mismo dado que tengo $k$ de capacidad disponible.
Esta noción, junto con la idea de maximizar el valor total para la capacidad disponible nos da una idea de que es lo que tenemos que resolver en cada subproblema. \\

Si no ponemos el objeto, entonces deberemos ver que es lo mejor que se hizo con los objetos anteriores para esa misma capacidad, si no, al utilizar el objeto, tenemos que ver que es lo mejor que se logró sumando el valor del objeto actual $valor(e_i)$ al valor dado por el resultado de lo mejor que se pudo lograr con los objetos anteriores para una capacidad de mochila disponible menor, la cual será $k$ - $peso(e_i)$ siempre que el peso del objeto sea menor que la capacidad disponible, dado que si no, estamos obligados a quedarnos con lo mejor que se pudo realizar con los anteriores $i-1$ objetos y la misma capacidad.

Luego, como los resultados a los subproblemas están bien definidos y son óptimos, el resultado para $S(i, k)$ será quedarse con el mejor de los dos.

Lo que nos queda es un máximo que se calcula de la siguiente manera:

\begin{equation}
S(i, k) = max(S(i-1, k-peso(e_i))+valor(e_i), S(i-1, k))
\end{equation}

La solución a nuestro problema será el resultado de resolver $S(n, K)$, siendo $n$ la cantidad de objetos total, la cual será óptima para $n$ objetos.\\

Como puede verse, en cada paso, para un objeto se puede determinar que acción realizar, con esta misma idea y con la ayuda de una matriz que nos permite hacer uso de memorizaci\'on de cada paso, podemos obtener que objetos estuvieron involucrados en el \'optimo final.\\

Para el caso de dos o tres mochilas estamos ante un problema de similar resolución, pero con más complicaciones relacionadas al logro de memorización para no perder información de cada paso realizado, la cual es fundamental para la obtención de los objetos de cada mochila.\\

Aquí la idea es obtener un máximo general sin importar si las mochilas maximizan o no por su parte.

Por cada objeto tendremos que decidir si no lo usamos, o en cual de las mochilas disponibles nos conviene ingresarlo y esas son todas las opciones disponibles para un objeto.\\
Para simplificar la notación, utilizaremos el caso de dos mochilas, aunque lo que sigue es facilmente adaptable a tres.

Como mencionamos el subproblema que analizamos es:

\begin{center}
$S(i, k_1, k_2)$ = máximo beneficio obtenido con $i$ objetos dado que dispongo de dos capacidades $k_1$ y $k_2$
\end{center}

Si se ingresa el objeto $i-esimo$ en la mochila $j-esima$, entonces tendremos que tomar lo mejor realizado con $i-1$ objetos y capacidad $k_j-peso(e_i)$ y las capacidades restantes sin modificar siempre y cuando $k_j \geq peso(e_i)$.
Si no se ingresa el objeto, se toma lo mejor realizado con $i-1$ objetos y todas las capacidades disponibles.\\

Como cada subproblema está bien definido y es óptimo, solo se tiene que elegir el mejor de todos ellos mediante un máximo. 

\begin{equation}
S(i, k_1, k_2) = 
\end{equation}
\begin{equation}
max(S(i-1, k_1-peso(e_i), k_2)+valor(e_i), 
        S(i-1, k_1, k_2-peso(e_i))+valor(e_i), 
        S(i-1, k_1, k_2))
\end{equation}

Por lo cual, la respuesta a este problema será $S(n, K_1, K_2)$ siendo $n$ la cantidad de objetos total y $K_1$ y $K_2$ las capacidades de las mochilas.\\

Además por lo explicado en la resolución del problema, tambi\'en se hace uso de matrices para resolver los casos con dos y tres mochilas, pero cabe destacar el plural. \\

En un principio, para obtener el máximo, podr\'iamos utilizar una única matriz. Pero dado que para cada objeto se la recorre completa, la información del máximo logrado para cada objeto puede no mantenerse, y además como los objetos no ocupan un lugar en la matriz, no podemos saber cual de estos lo logró.\\
Por lo tanto, hacemos uso de $n$ matrices de dimensiones correspondientes a la capacidad de las mochilas involucradas. \\

De esta manera, el máximo logrado para $i$ objetos estará alojado en la matriz correspondiente al objeto $i-esimo$.\\

De aquí en más, lo visto en la explicación ayuda a obtener los objetos en los casos para una y dos o tres dimensiones, teniendo en cuenta que es lo que sucede o no al tomar un objeto. \\

Por lo tanto,  podemos asegurar que el máximo logrado es el \'optimo y que los objetos pueden obtenerse como una consecuencia de las elecciones realizadas y la información correctamente almacenada.
