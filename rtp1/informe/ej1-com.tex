\vspace*{1em}

Realizando pasos 2-1, cada solución construída tendrá {\bf a lo sumo n-1 pasos}, siendo n la cantidad total de personas en la comitiva.

En el primer paso se analizan $\binom {n}{2}$ parejas posibles a enviar y por cada una de ellas se eligirá de entre 2 personas para ser el farlero.\\

En el segundo paso se tendr\'an $\binom {n-1}{2}$ parejas posibles y 3 faroleros para elegir.\\

Para el (i)-esimo paso se tendrán $\binom {n-(i-1)}{2}$ parejas e i+1 faroleros.\\

Por cada una de las posibilidades de cada paso, se deber\'an analizar la cantidad total de
posibilidades que haya en el paso subsiguiente, para dar lugar as\'i al analisis de todas las combinaciones posibles, es decir: siendo que el segundo paso se ejecuta $2 \ast \binom {n}{2}$ veces, el tercero $2 \ast \binom {n}{2} \binom{n-1} {2} \ast 3$, podemos ver que en combinación, todos los pasos demandarían:
\[
\prod_{i=1}^{n-1}\binom {n-(i-1)}{2}*(i+1) = \prod_{i=1}^{n-1}\binom {n-(i-1)}{2} * \prod_{i=0}^{n-1}(i+1)
\]

\[
n!\prod_{i=0}^{n-2}\binom {n-i}{2} = n!\prod_{i=0}^{n-2}\frac{(n-i)(n-(i-1))}{2}
\]

\[
\frac{ n!}{2^{n-2}}.\prod_{i=0}^{n-2}(n-i).\prod_{i=0}^{n-2}(n-(i-1)) = \frac{n!}{2^{n-2}}.n!.\frac{(n+1)!}{2}\leq \frac{n!^{2}.(n+1)!}{2^{n-1}}\in O(n.n!^{3})
\]

Para analizar la complejidad temporal consideraremos un árbol de opciones que en algoritmo recorre, en donde cada caso copleto evaluado, sea solución o no está caracterizado por una hoja y la rama que va desde la raíz a cada una de ellas es la secuencia de pasos necesarios para alcanzar cada caso.

Al ver las distintas podas efectuadas en el algoritmo podemos ver que la referida al tiempo no caracteriza ola complejidad del mismo: pude darse el caso en que la mejor solución se encuentre en la primer rama evaluada, con lo cual todas las demas serán analizadas de forma acotada gracias a la poda; o bien puede darse el caso en que el orden de las ramas evaluadas sea de mayor a menor en tiempo de solucion, con lo que se busca obligatoriamente sin aplicar la poda en ningún momento.

La poda referida a la cantidad de opciones a tener en cuenta por viaje a travez del puente, de las $N \times N$ combinaiones existentes solo toma $\binom {n}{2}$ tuplas de personas y $n$ personas solas para evaluar, es decir $O(\binom {n}{2} + n) \subseteq O(\binom {n}{2})$ posibilidades....To be continued