\indent Para corroborar el correcto funcionamiento de nuestro algoritmo implementado desarrollamos los siguientes tests:\\

Dentro de los casos en los cuales las capacidades de las mochilas pueden ser o no id\'enticas y los objetos son iguales o no, pudimos separar ciertas familias de casos puntuales:\\

\begin{center}
\textbf{Caso 1: No entra ning\'un objeto en las mochilas}
\end{center}

Este caso se da cuando $P_{i} > K_{j}$ con, 1 $\leq$ i $\leq$ N y 1 $\leq$ j $\leq$ 3\\

\begin{center}
\textbf{Caso 2: Entra un \'unico objeto en cada mochila}

\end{center}

Este caso se da cuando existe un \'unico objeto para cada mochila que presenta un peso menor a la capacidad de las mismas

\begin{center}
\textbf{Caso 3: Entra un \'unico objeto en total}

\end{center}

Este caso se da cuando existe un \'unico objeto que entra en alguna de las 3 mochilas, dejando as\'i dos mochilas totalmente vacias.

\begin{center}
\textbf{Caso 4: Entra una cantidad par de objetos en las mochilas}

\end{center}

\begin{center}
\textbf{Caso 5: Entra una cantidad dispar de objetos en las mochilas}

\end{center}

Este caso, tambien lo denominaremos random ya que existir\'a la posibilidad que entren de 0 a n objetos en las mochilas

\begin{center}
\textbf{Caso 6: Entran todos los objetos en las mochilas}
\end{center}

Este caso se da cuando todos los objetos ingresan en las mochilas de alguna manera \'optima, obteniendo as\'i la cantidad m\'axima posible de objetos la cual ser\'a la totalidad de los mismos

