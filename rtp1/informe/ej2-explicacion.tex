
Para solucionar el problema de poder quitar la llave y dejar equilibrada la balanza, realizamos un algoritmo que explicaremos a continuaci\'on:\\

Una aclaraci\'on previa que ser\'a de utilidad: como se dio como precondici\'on que la llave puede llegar a tomar un valor limite de $10^{15}$, trabajaremos con variables del tipo long long las cuales nos permitir\'an alcanzarlo.\\

$1)$ Obtendremos para empezar, el valor de la pesa que sea igual o mayor al peso de la llave $P$.
Sabemos que si hay una pesa con el mismo peso que la llave, es decir, si la llave $P$ tiene peso potencia de 3, nuestro algoritmo puede dar por finalizado su proceso y solo deberemos colocar esa pesa en el balanzin donde se encontraba la llave $P$ para lograr el equlibrio deseado.

Por otro lado, deberemos continuar para poder encontrar las pesas adecuadas.\\

$2$ Como las pesas, presentan un peso potencia de 3, creamos un arreglo de sumas parciales $sumasParciales$ donde iremos guardando las sumas parciales de las potencias desde $3^0$ hasta un $3^i$ en el paso $i-esimo$. La última suma parcial será hasta la potencia mayor que el valor que representa el peso de la llave ($P$) encontrado en el paso $1)$.\\

Cada valor en el arreglo $sumasParciales$ representa un intervalo 

Finalizado esto, realizaremos una b\'usqueda binaria para llevar el valor de p a 0. Para un trabajo m\'as sencillo guardamos el valor inicial de $P$ en la variable $equilibrioActual$ a la cual le iremos restando y/o sumando el valor de nuestras pesas.\\

Dicha b\'usqueda binaria la realizaremos en un ciclo que ir\'a desde el valor en m\'odulo de $equilibrioActual$ e iteraremos el mismo hasta que sea 0. Luego, como en toda b\'usqueda binaria, trabajaremos con nuestro array $sumasParciales$ chequeando si en la mitad del array nuestra $sumaParcial$ es mayor o igual al valor en m\'odulo de $equilibrioActual$. En caso de que fuese verdadero, chequeamos si el valor de $equilibrioActual$ es mayor o menor a 0. Si es menor a 0, sumaremos nuestra pesa correspondiente al \'indice en el que estamos de nuestro array $sumasParciales$, al valor de $equilibrioActual$ y guardaremos nuestra pesa en el $arrayD$ que simboliza al plato derecho de la balanza. Si es mayor a 0, en vez de sumarla la restamos y la guardamos en el array $arrayI$ que simboliza el otro plato. Siguiendo el razonamiento de la b\'usqueda binaria, volveremos a partir nuestro array en dos y haremos el mismo chequeo.\\

En caso de que el valor en m\'odulo de $equilibrioActual$ sea mayor que la mitad de $sumasParciales$ nos quedaremos con la mitad m\'as grande del arreglo e iteraremos nuevamente.\\

Una vez que llegamos a 0 y por consiguiente salimos de dicho ciclo, tendremos nuestras pesas ordenadas de mayor a menor en $arrayD$ y en $arrayI$, bastar\'a con invertir los arreglos para que queden de menor a mayor y devolver los mismos, finalizando as\'i nuestro algoritmo.\\
