En nuestro algoritmo como hemos mencionado anteriormente en la explicaci\'on del mismo, la etapa m\'as importante es a la hora de obtener las pesas que equilibran la balanza, en donde se realiza un ciclo que va disminuyendo el valor $equilibrioActual$ (inicialmente $P$) hasta llegar a 0 recorriendo el arreglo de $sumasParciales$ de sumas de potencias de tres.\\
Una vez alcanzado ese valor se puede dar por finalizado el algoritmo, dado que hemos encontrado una manera de sumar y restar potencias de tres tales que como resultado se obtiene $P$.\\

Por lo tanto para probar que el algoritmo es correcto deberemos probar primero que el algoritmo termina, es decir, que $equilibrioActual$ alcanza el valor cero luego de iterar el arreglo $sumasParciales$ una sola vez y luego que utilizamos siempre potencias de tres diferentes.\\

Queremos probar entones que nuestro algoritmo encuentra las pesas necesarias para todo $P$ menor que alguna suma de potencias en modulo, es decir:

\begin{equation}
(\forall P \in \mathbb{Z}), |P| \leq \sum_{i=0}^{k}(3^i) \textit{ con k $\geq$ 0}
\end{equation}

El algoritmo encuentra una combinación de sumas y restas de potencias de tres que es igual a $P$. Podemos utilizar inducción.\\

Para $k$ = 0, $P$ tiene que ser igual a uno en modulo. Por lo cual, luego de realizar la resta el algoritmo ha finalizado.\\

Para el paso inductivo $k = n$ sabemos que nustro algoritmo encuentra la repuesta:

\begin{equation}
(\forall P \in \mathbb{Z}), |P| \leq \sum_{i=0}^{n}(3^i) \textit{ con n > 0}
\end{equation}

Sea $P^{'}$  tal que: 

\begin{equation}
\sum_{i=0}^{n}(3^i) <  |P^{'}| \leq \sum_{i=0}^{n+1}(3^i) 
\end{equation}

el algoritmo realiza el siguiente operación:

\begin{equation}
P^{''} = \left\{ \begin{array}{lcc}
             3^{n+1} + P^{'} & \textit{con $P^{'} <$ 0} \\
              P^{'} - 3^{n+1} & \textit{con $P^{'} >$ 0} 
             \end{array}
             \right.
\end{equation} 

Veamos que en cualquier caso, el nuevo valor $P^{''}$ es menor a $\sum_{i=0}^{n}(3^i)$.

En el caso $P^{'} <$ 0:

\begin{equation}
3^{n+1} - |P^{'}| < \sum_{i=0}^{n}(3^i) \iff
|P^{'}| > - \sum_{i=0}^{n}(3^i) + 3^{n+1}
\end{equation}

Lo cual es cierto ya que:

\begin{equation}
- \sum_{i=0}^{n}(3^i) + 3^{n+1} >  \sum_{i=0}^{n}(3^i) \iff
3^{n+1} > 2 \ast \sum_{i=0}^{n}(3^i) \iff
\end{equation}

\begin{equation}
3^{n+1} > 2 \ast \frac{1-3^{n+1}}{1-3} \iff
3^{n+1} > -1+3^{n+1} \iff
0 > -1
\end{equation}

El caso con $P^{'} >$ 0 se puede ver directamente pasando $3^{n+1}$ sumando.

\begin{equation}
P^{'} - 3^{n+1} < \sum_{i=0}^{n}(3^i) \iff
P^{'} < \sum_{i=0}^{n}(3^i) + 3^{n+1} \iff
P^{'} < \sum_{i=0}^{n+1}(3^i)
\end{equation}

Por lo cual, el valor de $P^{''}$ no se encuentra en el mismo intervalo que $P^{'}$ y como $3^{n+1}$ ya no se encuentra en intervalos más pequeños, el intervalo que contenga a $P^{'}$ utilizará potencias diferentes. Como por hipótesis inductiva podemos generar cualquier $P$ mayor que cero, podremos en particular generar a $P^{''}$.

Luego el algoritmo finaliza y además obtiene todas las pesas adecuadas.\\
