En nuestro algoritmo como hemos mencionado anteriormente en la explicaci\'on del mismo, la etapa m\'as importante es a la hora de obtener las pesas que equilibran la balanza, en donde se realiza un ciclo que va disminuyendo el valor $equilibrioActual$ (inicialmente $P$) hasta llegar a 0 recorriendo el arreglo de $sumasParciales$ de sumas de potencias de tres.\\
Una vez alcanzado ese valor se puede dar por finalizado el algoritmo, dado que hemos encontrado una manera de sumar y restar potencias de tres tales que como resultado se obtiene $P$.\\

Por lo tanto para probar que el algoritmo es correcto deberemos probar primero que el algoritmo termina, es decir, que $equilibrioActual$ alcanza el valor cero luego de iterar el arreglo $sumasParciales$ una sola vez y luego que utilizamos siempre potencias de tres diferentes.\\

En esta parte de la demostración no nos interesa mantener el signo del resto, dado que saber luego si la pesa tomada resta o suma es una consecuencia de tener que alcanzar el valor cero chequeando en cada paso si el equilibrio conseguido es mayor o menor que cero respectivamente y por lo tanto tomaremos el modulo de cada resto.\\

Queremos probar entones que nuestro algoritmo encuentra las pesas necesarias:

\begin{equation}
\forall P \leq \sum_{i=0}^{k}(3^i) \textit{con k $\geq$ 0}
\end{equation}

Podemos utilizar inducción.\\

Para $k$ = 0, tenemos que tener un $P$ menor o igual a uno. Si es cero ya hemos finalizado y si es uno podemos restar uno y esa será la pesa que debemos agregar. Como el resto es cero, ya hemos finalizado.\\

Para el paso inductivo $k = n$ sabemos que nustro algoritmo encuentra la repuesta:

\begin{equation}
\forall P \leq \sum_{i=0}^{n}(3^i) \textit{con n > 0}
\end{equation}

Sea $P^{'}$  tal que: 

\begin{equation}
\sum_{i=0}^{n}(3^i) <  P^{'} \leq \sum_{i=0}^{n+1}(3^i) 
\end{equation}

el algoritmo realiza el siguiente operación:

\begin{equation}
P^{''} = | 3^{n+1} - P^{'} |
\end{equation} 

Veamos que al restar $P^{'}$ por $3^{n+1}$, el nuevo valor $P^{''}$ es menor a $\sum_{i=0}^{n}(3^i)$.

\begin{equation}
P^{'} < \sum_{i=0}^{n+1}(3^i) \iff
P^{'} < \sum_{i=0}^{n}(3^i) + 3^{n+1} \iff
P^{'} - 3^{n+1} < \sum_{i=0}^{n}(3^i) \iff
P^{''}  < \sum_{i=0}^{n}(3^i)
\end{equation}

Por lo cual, el valor de $P^{''}$ no se encuentra en el mismo intervalo que $P^{'}$ y como $3^{n+1}$ ya no se encuentra en intervalos más pequeños, el intervalo que contenga a $P^{'}$ utilizará potencias diferentes. Como por hipótesis inductiva podemos generar cualquier $P$ mayor que cero, podremos en particular generar a $P^{''}$.

Luego el algoritmo finaliza y además obtiene todas las pesas adecuadas.\\
