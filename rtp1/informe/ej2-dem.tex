En nuestro algoritmo como hemos mencionado anteriormente en la explicaci\'on del mismo, la etapa m\'as importante es, a la hora de obtener las pesas que equilibran la balanza, donde se realiza un ciclo que va disminuyendo el valor $equilibrioActual$ (inicialmente $P$) hasta llegar a 0 recorriendo el arreglo de $sumasParciales$ de sumas de potencias de tres.\\
Una vez alcanzado ese valor se puede dar por finalizado el algoritmo, dado que hemos encontrado una manera de sumar y restar potencias de tres tales que como resultado se obtiene $P$.\\

Por lo tanto para probar que el algoritmo es correcto deberemos probar primero que el algoritmo termina, es decir, que $equilibrioActual$ alcanza el valor cero luego de iterar el arreglo $sumasParciales$ una sola vez y luego que utilizamos siempre potencias de tres diferentes.\\

Queremos probar entonces que nuestro algoritmo encuentra las pesas necesarias para todo $P$ $ \in \mathbb{Z}$ utilizando a lo sumo $k$ potencias de tres diferentes, con $k$ el primer valor tal que la suma de las primeras $k$ potencias es mayor o igual a $P$. Es decir:

\begin{equation}
(\forall P \in \mathbb{Z}), |P| \leq \sum_{i=0}^{k}(3^i) \textit{ con k $\geq$ 0}
\end{equation}

Para $k$ = 0, $P$ tiene que ser igual a uno en modulo. Por lo cual, luego de realizar la resta el algoritmo ha finalizado.\\

Para el paso inductivo $k = n > 0$ sabemos que nuestro algoritmo encuentra la repuesta a cualquier $P$ con a lo sumo $n$ potencias de tres diferentes tal que $n$ es el primer valor que cumple que la suma de las primeras $n$ potencias es mayor o igual a $P$.

Sea $P^{'}$ tal que $n+1$ es el primer valor que cumple que la suma de las primeras $n+1$ potencias es mayor o igual a $P^{'}$, entonces:

\begin{equation}
\sum_{i=0}^{n}(3^i) <  |P^{'}| \leq \sum_{i=0}^{n+1}(3^i) 
\end{equation}

El algoritmo realiza la siguiente operación:

\begin{equation}
P^{''} = \left\{ \begin{array}{lcc}
             3^{n+1} + P^{'} & \textit{con $P^{'} <$ 0} \\
              P^{'} - 3^{n+1} & \textit{con $P^{'} >$ 0} 
             \end{array}
             \right.
\end{equation} 

Veamos que en cualquier caso, el nuevo valor $P^{''}$ es menor o igual a $\sum_{i=0}^{n}(3^i)$.

En el caso $P^{'} <$ 0:

\begin{equation}
3^{n+1} - |P^{'}| \leq \sum_{i=0}^{n}(3^i) \iff
|P^{'}| \geq - \sum_{i=0}^{n}(3^i) + 3^{n+1}
\end{equation}

Lo cual es cierto ya que:

\begin{equation}
- \sum_{i=0}^{n}(3^i) + 3^{n+1} >  \sum_{i=0}^{n}(3^i) \iff
3^{n+1} > 2 \ast \sum_{i=0}^{n}(3^i) \iff
\end{equation}

\begin{equation}
3^{n+1} > 2 \ast \frac{1-3^{n+1}}{1-3} \iff
3^{n+1} > -1+3^{n+1} \iff
0 > -1
\end{equation}

Además veamos que $- \sum_{i=0}^{n}(3^i) + 3^{n+1}$ $>$ $|P^{'}|$ no puede suceder dado que el 

m\'odulo del intervalo es:

 \begin{equation}
- \sum_{i=0}^{n}(3^i) + 3^{n+1} - \sum_{i=0}^{n}(3^i) = 
 3^{n+1} - 2 \ast \sum_{i=0}^{n}(3^i) =
 3^{n+1} - (-1 + 3^{n+1}) = 
1 
\end{equation}

Y $P$ es entero, por lo cual nunca tomará valores decimales.\\


El caso con $P^{'} >$ 0 se puede ver directamente pasando $3^{n+1}$ sumando.

\begin{equation}
P^{'} - 3^{n+1} \leq \sum_{i=0}^{n}(3^i) \iff
P^{'} \leq \sum_{i=0}^{n}(3^i) + 3^{n+1} \iff
P^{'} \leq \sum_{i=0}^{n+1}(3^i)
\end{equation}

Por lo cual, el valor de $P^{''}$ se encuentra en un intervalo m\'as pequeño que $P^{'}$. Por hipótesis inductiva, podemos generar a $P^{'}$ con a lo sumo $n$ potencias de tres diferentes, y como la pesa más grande será $3^{n}$, se garantiza que no se repita $3^{n+1}$.
Por lo tanto, hemos probado inductivamente que vale (5). Es decir, podemos generar cualquier numero entero, con a lo sumo $n$ potencias de tres diferentes, con $n$ el primero que cumple la propiedad enunciada en (5).
Como en a lo sumo $k$ iteraciones se llega al valor cero, el algoritmo finaliza y se obtiene la solución al problema.
