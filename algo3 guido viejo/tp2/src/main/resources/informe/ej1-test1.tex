\indent Por consiguiente, mostraremos buenos y malos casos para nuestro algoritmo, y a su vez, daremos el tiempo estimado 
seg\'un la complejidad del algoritmo calculada anteriormente.\\

Para llegar a dicha conclusi\'on trabajamos con un total de 100 instancias y un n entre 1 y 1000000 obtuvimos que nuestro
algoritmo finaliza lo solicitado demorando 184 milisegundos.\\

Para una mayor observacion desarrollamos el siguiente grafico con las instancias:\\


Si a esto lo dividimos por la complejidad propuesta obtenemos:\\

 Para realizar esta divisi\'on realizamos un promedio con el mismo input de aproximadamente 20 corridas
tanto para la complejidad como para nuestro algoritmo y una vez calculado dicho promedio de ambas cosas realizamos la divisi\'on para
obtener resultados m\'as consisos.\\ 

  
  
A continuaci\'on, adjuntamos una tabla con los considerados “mejor” caso que nos parecieron m\'as relevantes

Dando un \textbf{promedio igual a }\\

Para llegar a dicha conclusi\'on trabajamos con un total de 100 instancias y un n entre 1 y 1000000 obtuvimos que nuestro
algoritmo finaliza lo solicitado demorando 224 milisegundos.\\


Si a esto lo dividimos por la complejidad propuesta obtenemos:\\

  Para realizar esta experimentaci\'on nos parecio acorde, realizar un promedio con el mismo input de aproximadamente 20 corridas
tanto para la complejidad como para nuestro algoritmo y una vez calculado dicho promedio de ambas cosas realizamos la divisi\'on para
obtener resultados m\'as relevantes.\\ 


La informaci\'on de los 10 datos mas relevantes referiendonos al peor caso fueron:

Dando un \textbf{promedio igual a } \\

Aqu\'i, podemos observar como la cota de complejidad del algoritmo y la de dicho caso tienden al mismo valor con el paso del tiempo.\\