La mediana de un conjunto ordenado de $n$ n\'umeros se define como $x_{(n+1)/2}$ si $n$ es impar, o como
$\frac{1}{2}(x_{n/2} + x_{n/2+1})$ si $n$ es par. Dados $n$ n\'umeros enteros en cualquier orden se deben devolver otros $n$
n\'umeros, donde el $i$-\'esimo de ellos represente la parte entera de la mediana de los primeros $i$ n\'umeros de
la entrada.\\\\
Resolver en una complejidad estrictamente mejor que O($n^2$) donde $n$ es el n\'umero total de enteros de
entrada\\\\
\textbf{Entrada Tp1Ej2.in} y \textbf{Salida Tp1Ej2.out}\\
Tendr\'an una sucesi\'on de enteros separados por espacios a raz\'on de una instancia del problema por l\'inea
respectivamente.\\\\
PISTA: Mantener luego de cada entero le\'ido de la entrada el conjunto actual dividido en dos mitades, los
m\'as chicos y los m\'as grandes. Considerar una estructura de datos eficiente para manipular esos conjuntos.\\