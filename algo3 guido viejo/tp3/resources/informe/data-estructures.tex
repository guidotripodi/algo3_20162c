\subsection{Graph}

Para mayor facilidad, implementamos una estructura que habla del estado del grafo que esta siendo coloreado. Esta estructura brinda la siguiente información:
\begin{itemize}
\item Para un nodo indica las opciones de coloreo que no generan conflictos.
\item Devuelve la cantidad de aristas con sus dos extremos pintados y de distinto color
\item Indica el color actual para cada nodo
\item Indica si un nodo esta coloreado
\end{itemize}

A su vez, dicha estructura admite las siguientes operaciones:
\begin{itemize}
\item Pintar un nodo del "mejor color". Donde por "mejor color" nos referimos al color que genera menos conflictos entre sus vecinos ya coloreados. Esta operación, además se encarga de modificar el color actual del nodo dado, y también modifica las restricciones de sus vecinos.
\end{itemize}	

\subsection{ValueSortedMap}
Fue necesario implementar una estructura que funcione como diccionario (a cada clave asignarle un valor) y que a su vez pueda devolver la clave con el mayor y el menor valor. Es decir, nuestra estructura soporta la siguientes operaciones (complejidades en función del tamaño $N$):
\begin{itemize}
\item Insertar una clave y su valor. (Si la clave ya existe, actualizar su valor). Costo $O(log(N))$.
\item Elimina y devuelve la clave de mayor o menor valor. Costo $O(log(N))$
\item Devuelve la clave de mayor o menor valor. Costo $O(log(N))$
\item Indica su tamaño. Costo $O(1)$
\end{itemize}

Para la implementación utilizamos dos diccionarios, uno con las entradas agrupadas por clave y otro con las entradas agrupadas por valor. La complejidad de las operaciones de nuestra estructura se basa en que las implementaciones de TreeSet y TreeMap (Arboles binarios balanceados) las operaciones de busqueda, agregado y eliminación toman tiempo logaritmico.
	 