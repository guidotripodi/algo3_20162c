\indent Por consiguiente, mostraremos algunos casos particulares para nuestro algoritmo.\\

%\textbf{Usar un único portal que lleva al piso N}\\
\\
A continuación mostraremos un gráfico con el tiempo de ejecución de nuestro algoritmo y la cota de complejidad.\\

\vspace*{0.3cm} \vspace*{0.3cm}
  \begin{center}
% \includegraphics[scale=0.5]{./ej1/ej1_mejorcaso.png}
  \end{center}
  \vspace*{0.3cm}


S%e puede observar como el tiempo de ejecución del algoritmo en promedio se encuentra entre las funciones de calcular el tiempo para $N^2$ y $2.N^2$ operaciones, demostrando como nuestra complejidad calculada es la correcta.\\

  
%\textbf{Usar N portales para subir N pisos}\\

A continuación mostraremos un gráfico para ver como se comporta nuestro algoritmo con relación a la cota de complejidad en referencia a este caso.

\vspace*{0.3cm} \vspace*{0.3cm}
  \begin{center}
 %\includegraphics[scale=0.5]{./ej1/ej1_peorcaso.png}
  \end{center}
  \vspace*{0.3cm}
  
%Se puede observar como en este caso nos encontramos en el medio de las funciones, afirmando que la complejidad calculada es $O(N^2)$.
