Estamos en el año 2048 y el Pabell on 0+infinito es todo un  éxito. Los alumnos de Algoritmos III están
contentos porque van a cursar este cuatrimestre en un aula que esta en el piso N, que es el más alto de
todos. Con los avances de la ciencia y la tecnología, escaleras y ascensores han quedado obsoletos, y la
forma de subir de un piso a otro es a traes de portales. El nuevo pabellon tiene P portales, cada uno
de los cuales permite subir de un piso A a un piso mas alto B (para bajar de piso hay que tirarse con
paracaıdas al piso 0 y luego volver a subir de ser necesario). Uno de los alumnos, que estaba cursando en
el segundo cuatrimestre de 2015 y fue congelado por el metodo de criogenia, acaba de ser descongelado
y no puede creer lo buenos que estan estos portales, algo que en su época no existía. Luego de completar
todos los censos de estudiantes desde el año 2016 en adelante, este alumno quiere usar la mayor cantidad
de portales posibles para llegar al piso N y asi seguir cursando Algoritmos III. Diseñar un algoritmo de
complejidad O($N^2$) para calcular la mayor cantidad de portales que puede utilizar el alumno para subir
desde planta baja al piso N (sin tirarse nunca con paracaıdas). Se asegura que en toda instancia del
problema es posible realizar el recorrido deseado, y que no hay más de un portal que comunique el mismo
par de pisos.