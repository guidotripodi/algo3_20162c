\indent Por medio de los tests dados por la c\'atedra, desarrollamos nuestros tests,
para corroborar que nuestro algoritmo era el indicado.\\

A continuaci\'on enunciaremos los tipos de casos diferentes que encontramos con mayor relevancia a la hora de realizar
testeos sobre el algoritmo:\\

\begin{center}
 \textbf{Unico portal que lleva a un unico piso N}
\end{center}
 Para este tipo de testeo mostraremos a continuaci\'on un ejemplo del mismo, exponiendo su respectivo resultado.\\

 Con un:\\\\
  \indent  $Piso:$ $10$
  
  Obtuvimos el siguiente resultado:\\
  
  $Cantidad$ $de$ $portales$ $usados:$ $1$


 \begin{center}
 \textbf{Portales que pueden conectar todos los pisos desde el 0 hasta el N}
\end{center}
 Como ejemplo mostraremos, exponiendo su respectivo resultado, el siguiente testeo.\\

 Con un secuencia de pisos:\\\\
  \indent  $Piso:$ $0 10 20 $
  
  Obtuvimos el siguiente resultado:\\
  
  $Cantidad$ $de$ $portales$ $usados:$ $3$
  
  
 
 \begin{center}
 \textbf{Ning\'un portal posible que conecte una secuencia para llegar al piso N}
\end{center}
 Como ejemplo mostraremos, exponiendo su respectivo resultado, el siguiente testeo.\\

 Con un secuencia de pisos:\\\\
  \indent  $Piso:$ $0 4; 1 5; 2 6; 5 9$
  
  Obtuvimos el siguiente resultado:\\
  
  $Cantidad$ $de$ $portales$ $usados:$ $0$