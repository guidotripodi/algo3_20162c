El Pabellon 0+infinito, nuevamente remodelado, ahora tiene un diseño basado en un conjunto de M
pasillos de distintas longitudes con intersecciones en donde se unen dos o mas pasillos. Es ası que puede
modelarse como un grafo con pesos en los ejes, donde cada eje es un pasillo (de peso igual a la longitud del
pasillo), y cada vértices es una intersección o un extremo donde termina un pasillo sin unirse con ningun
otro. El decano junto con el director del Departamento de Computacion, están preocupados porque talvez
existen ciclos en dicho grafo, lo que podría perjudicar a los alumnos al hacer que se pierdan buscando
las aulas. Por tal motivo el decano decidió clausurar los pasillos que sea necesario de manera tal que no
queden ciclos en el grafo que representa al pabellon. El problema es que cuanto más largo es un pasillo,
más costoso es clausurarlo. 
Diseñar un algoritmo de complejidad O(M log M) para calcular la mínima
suma posible de las longitudes de los pasillos que deberían ser clausurados (eventualmente ninguno) para
que no existan ciclos formados por tres o mas pasillos en el grafo que representa al pabellón. Se asegura
que en toda instancia del problema el grafo que representa al pabellon es conexo.