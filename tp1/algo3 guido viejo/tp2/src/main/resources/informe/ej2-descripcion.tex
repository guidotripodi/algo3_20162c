El Pabell\'on 0+infinito acaba de reabrir sus puertas con la novedad de que ahora tiene P portales que
son bidireccionales; asimismo, los paracaıdas fueron eliminados por considerarse inseguros. Cada piso
del renovado pabell\'on consta de un pasillo de L metros de longitud, y cada portal permite viajar entre
posiciones espec\'ificas de los pasillos de dos pisos. M\'as concretamente, cada portal puede describirse por
medio de cuatro enteros no negativos A, DA, B y DB, los cuales indican que el portal comunica el
piso A, a DA metros del comienzo del pasillo de ese piso, con el piso B, a DB metros del comienzo
del pasillo de ese piso. Los alumnos, acostrumbrados a los portales que solo permitían subir, están un
poco confundidos al poder utilizar un mismo portal tanto para subir como para bajar entre dos pisos, o
incluso para moverse entre posiciones diferentes dentro del pasillo de un mismo piso. Todos los alumnos
de Algoritmos III quieren llegar primero a la clase, que es en un aula que está al final del piso N (el máas
alto del pabellón). \\
Diseñar un algoritmo de complejidad O(NL + P) para calcular la mínima cantidad
de segundos que se necesitan para llegar del comienzo del pasillo del piso 0 al final del pasillo del piso
N, suponiendo que recorrer un metro requiere 1 segundo, y utilizar cualquier portal requiere 2 segundos
(en cualquiera de las dos direcciones posibles). Se asegura que en toda instancia del problema es posible
realizar el recorrido deseado, y que no hay máas de un portal que comunique las mismas posiciones del
mismo par de pisos. No obstante, puede haber más de un portal que comunique el mismo par de pisos,
y portales que comuniquen posiciones diferentes dentro del pasillo de un mismo piso.