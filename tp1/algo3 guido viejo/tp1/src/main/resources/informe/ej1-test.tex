\indent Por medio de los tests dados por la c\'atedra, desarrollamos nuestros tests,
para corroborar que nuestro algoritmo era el indicado.\\

A continuaci\'on enunciaremos 4 tipos de casos de nuestros tests:\\

\begin{center}
 \textbf{Rollo de cable cubre todas las estaciones}
\end{center}
 Para este tipo de testeo mostraremos a continuaci\'on un ejemplo del mismo, exponiendo su respectivo resultado.\\

 Con un:\\\\
  \indent  $Rollo$ $de$ $Cable:$ $90$
  
  $Lista$ $de$ $estaciones:$ $80$ $81$ $82$ $83$ $84$ $85$ $86$ $87$ $88$ $89$ $90$\\
  
  Obtuvimos el siguiente resultado:\\
  
  $Cantidad$ $de$ $estaciones$ $conectadas:$ $11$

 \begin{center}
 \textbf{Rollo de cable no cubre ninguna de las estaciones}
\end{center}

Para este tipo de testeo mostraremos a continuaci\'on un ejemplo del mismo, exponiendo su respectivo resultado.\\

 Con un:\\
  \indent $Rollo$ $de$ $Cable:$ $60$

  $Lista$ $de$ $estaciones:$ $80$ $150$ $220$ $290$ $360$\\
  
  Obtuvimos el siguiente resultado:\\
  
  $Cantidad$ $de$ $estaciones$ $conectadas:$ $0$


\begin{center}
 \textbf{Rollo de cable con estaciones de igual o similar Kilometraje en referencia a la distancia}
\end{center}

Para este tipo de testeo mostraremos a continuaci\'on un ejemplo del mismo, exponiendo su respectivo resultado.\\

\indent $Rollo$ $de$ $Cable:$ 50\\
$Lista$ $de$ $estaciones:$ 50 60 69 70 130 190\\

Obtuvimos el siguiente resultado:\\
  
  $Cantidad$ $de$ $estaciones$ $conectadas:$ 4

\begin{center}
 \textbf{Estaciones con bastante distancia intercaladas con estaciones m\'as cercanas}
\end{center}

Aqu\'i veremos, un ejemplo del conjunto de test de este tipo, exponiendo su respectivo resultado.\\

\indent $Rollo$ $de$ $Cable:$ 200\\
$Lista de estaciones:$ 15 16 17  40 41 42 70 71 72 100 101 102 140 141 142 170 171 172 200 201 202 230 231 232\\

Obtuvimos el siguiente resultado:\\ 
  
  $Cantidad$ $de$ $estaciones$ $conectadas:$ 21
