¡La comunicaci\'on es el progreso! Decididos a entrar de lleno en la nueva era, el pa\'is decidi\'o conectar
telegr\'aficamente todas las estaciones del moderno sistema f\'erreo que recorre el pa\'is en abanico con origen
en la capital (el kil\'ometro 0). Por lo escaso del presupuesto, se ha decidido ofrecer cierta cantidad de
kil\'ometros de cable a cada ramal. Pero para maximizar el impacto en \'epocas electorales se busca lograr
conectar la mayor cantidad de ciudades con los metros asignados (sin hacer cortes en el cable)\\\\
Resolver cu\'antas ciudades se pueden conectar para cada ramal en O(n), con n la cantidad de estaciones
en cada ramal, y justificar por qu\'e el procedimiento desarrollado resuelve efectivamente el problema.\\\\
Entrada \textbf{Tp1Ej1.in}\\\\
Cada ramal ocupa dos l\'ineas, la primera contiene un entero con los kil\'ometros de cable dedicados al
ramal, y la segunda los kilometrajes de las estaciones en el ramal sin cosiderar el 0.\\\\
Salida \textbf{Tp1Ej1.out}\\\\
Para cada ramal de entrada, se debe indicar una l\'inea con la cantidad de ciudades conectables encontradas.\\