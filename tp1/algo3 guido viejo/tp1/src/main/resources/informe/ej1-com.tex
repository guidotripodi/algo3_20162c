\vspace*{1em}
\begin{enumerate}

\item\textbf{Solve}
\par 
Comenzamos por convertir el \textit{input} provisto (una lista de $N$ enteros tal que el lugar $i$-\'esimo es la distancia a la capital de la estaci\'on $i$-\'esima) por una lista $Differences[]$ tal que $Differences[i]$ es la distancia entre las estaciones $i$ y $i+1$, que tiene una complejidad $O(N)$.

En la implementaci\'on del algoritmo, se itera por cada elemento de la lista. En cada iteraci\'on, actualizar la suma y la cantidad de elementos actualmente considerados es $O(1)$.

Ahora bien, como tanto el \'indice izquierdo (implementado con el \textit{iterator}) siempre est\'a a izquierda del derecho (implementado con el \textit{foreach}), y ambos \'indices s\'olo se mueven hacia la derecha, tenemos que el \'indice izquierdo s\'olo se mueve a derecha a lo sumo $N$ veces. Y cada vez que movemos el \'indice izquierdo, actualizar la cantidad de elementos y la suma actual es $O(1)$. Adem\'as, actualizar el m\'aximo logrado hasta el momento, tambi\'en es $O(1)$.

Por lo tanto, en total realizamos $O(N)$ iteraciones para el \'indice derecho, m\'as $O(N)$ iteraciones para el \'indice izquierdo, y cada iteraci\'on de ambos \'indices es $O(1)$, lo que nos da una complejidad total de $O(N)$.

Calculemos el mejor y peor caso del algoritmo. Dado que las iteraciones del \'indice derecho se realizan siempre de $1$ a $N$, el mejor caso se va a dar cuando no sea necesario mover el \'indice izquierdo. Esto se da, por ejemplo, cuando la suma de todas las distancias es menor que la longitud del cable (el cable alcanza para conectar todas las estaciones).

An\'alogamente, es f\'acil ver que el peor caso se da cuando es necesario llevar al \'indice izquierdo hasta el final de la lista, pues en este caso la cantidad de iteraciones es m\'axima. Esto se puede dar, por ejemplo, cuando la longitud del cable es menor que $Differences[i]	$ $\forall i$. 

\end{enumerate}
