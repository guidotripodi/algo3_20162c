\indent Luego de realizar la implementaci\'on de nuestro algoritmo, desarrollamos tests,
para corroborar que nuestro algoritmo era el indicado.\\

A continuaci\'on enunciaremos varios de nuestros tests:\\

\begin{center}
 \textbf{El valor de entrada $P$ es de la forma ${3^i}$ para un i$\gets$[0, N] }
\end{center}
 Este caso se cumple cuando se recibe un P el cual al realizar nuestro primer ciclo que chequea cual es la potencia igual o mayor, termina siendo igual y de esta forma solo se itera una \'unica vez el segundo y tercer ciclo.
 
\begin{center}
 \textbf{El valor de entrada $P$ es de la forma \[
\sum_{i=1}^{n}3_{i}=P 
\]}
\end{center}

Veremos mas adelante que este caso ser\'a el peor a resolver ya que se iterar\'a la totalidad completa de elementos de nuestros arrays.
 
 

\begin{center}
 \textbf{El valor de entrada $P$ es de la forma ${3^i} + R$ para  (i, R) $\gets$[0, N]}
\end{center}

 Este caso se cumple cuando se recibe un P el cual al realizar nuestro primer ciclo que chequea cual es la potencia igual o mayor, termina siendo mayor y de esta forma se itera mas de una vez el segundo y tercer ciclo.

\begin{center}
 \textbf{El valor de entrada $P$ es impar}
\end{center}

 Este caso se cumple cuando se recibe un P el cual el mismo es de la forma $P MOD 2 = 1$.
 
 \begin{center}
 \textbf{El valor de entrada $P$ es par}
\end{center}

 Este caso se cumple cuando se recibe un P el cual el mismo es de la forma $P MOD 2 = 0$.