\indent Para corroborar el correcto funcionamiento de nuestro algoritmo implementado desarrollamos los siguientes tests:\\


\begin{center}
 \textbf{Mochilas con capacidades id\'enticas y objetos distintos}
\end{center}

Este caso se da cuando $K_{i} = K_{j}$ con i $\neq$ j , 1 $\leq$ i $\leq$ 3  1 $\leq$ j $\leq$ 3  \\

A continuaci\'on un ejemplo de este tipo de caso:

Con:

M = 3 N = 3

$K_{1}$ = 15  $K_{2}$ = 15  $K_{3}$ = 15 

$C_{1}$ = 3 $P_{1}$ = 5 $V_{1}$ = 10

$C_{2}$ = 2 $P_{1}$ = 3 $V_{1}$ = 5

$C_{3}$ = 5 $P_{1}$ = 4 $V_{1}$ = 2
  
  \indent  
  
  Obtuvimos el siguiente resultado:\\

$Valor$ $total$ = 50

$Mochila1$ = 3 $Objetos:$ 1 1 1

$Mochila2$ = 4 $Objetos:$ 2 2 3 3

$Mochila3$ = 3 $Objetos:$ 3 3 3

$Cantidad$ $de$ $Objetos$ $totales:$ 10



\begin{center}
 \textbf{Objetos id\'enticos con mochilas distintas}
\end{center}

Este caso se da cuando $C_{i} = N$ con i = 1 y $K_{h}$ $\neq$ $K_{j}$ con h $\neq$ j , 1 $\leq$ h $\leq$ 3  1 $\leq$ j $\leq$ 3 \\

A continuaci\'on un ejemplo de este tipo de caso:\\

 Con:
 
 M = 3 N = 1
 
 $K_{1}$ = 5  $K_{2}$ = 10  $K_{3}$ = 15
 
 $C_{1}$ = 10 $P_{1}$ = 5 $V_{1}$ = 10
  
  \indent  
  
  Obtuvimos el siguiente resultado:\\

$Valor$ $total$ = 60

$Mochila1$ = 1 $Objetos:$ 1

$Mochila2$ = 2 $Objetos:$ 1 1

$Mochila3$ = 3 $Objetos:$ 1 1 1

$Cantidad$ $de$ $Objetos$ $totales:$ 6

\begin{center}
 \textbf{Mochilas distintas con objetos distintos}
\end{center}

Este caso se da cuando \[
\sum_{i=1}^{x}C_{i}=N 
\]  y $K_{h}$ $\neq$ $K_{j}$ con h $\neq$ j , 1 $\leq$ h $\leq$ 3  1 $\leq$ j $\leq$ 3 \\

A continuaci\'on un ejemplo de este tipo de caso:\\

 Con:
 
 M = 3 N = 3
 
 $K_{1}$ = 6  $K_{2}$ = 12  $K_{3}$ = 15
 
 $C_{1}$ = 3 $P_{1}$ = 2 $V_{1}$ = 8
 
 $C_{2}$ = 2 $P_{1}$ = 5 $V_{1}$ = 5
 
 $C_{3}$ = 5 $P_{1}$ = 4 $V_{1}$ = 2
  
  \indent  
  
  Obtuvimos el siguiente resultado:\\

$Valor$ $total$ = 42

$Mochila1$ = 2 $Objetos:$ 1 3

$Mochila2$ = 3 $Objetos:$ 3 3 3

$Mochila3$ = 4 $Objetos:$ 1 1 2 2

$Cantidad$ $de$ $Objetos$ $totales:$ 9


\begin{center}
 \textbf{Mochilas con capacidades id\'enticas y objetos iguales}
\end{center}

Este caso se da cuando $K_{i} = K_{j}$ con i $\neq$ j , 1 $\leq$ i $\leq$ 3  1 $\leq$ j $\leq$ 3  y $C_{h} = N$ con h = 1\\

A continuaci\'on un ejemplo de este tipo de caso:

 Con:
 
 M = 3 N = 1
 
 $K_{1}$ = 15  $K_{2}$ = 15  $K_{3}$ = 15
 
 $C_{1}$ = 10 $P_{1}$ = 3 $V_{1}$ = 6
 
  \indent  
  
  Obtuvimos el siguiente resultado:\\

$Valor$ $total$ = 60

$Mochila1$ = 5 $Objetos:$ 1 1 1 1 1

$Mochila2$ = 5 $Objetos:$ 1 1 1 1 1

$Mochila3$ = 0 $Objetos:$ 

$Cantidad$ $de$ $Objetos$ $totales:$ 10