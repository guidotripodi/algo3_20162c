Para solucionar este problema, se debe ver la combinaci\'on de viajes entre lados del puente (lado A, origen y lado B, destino) que nos permiten pasar a todos los integrantes
del grupo, del lado A al B en el menor tiempo posible. Siendo esta busqueda una tarea exponencial, buscamos la forma de poder disminuir los casos evaluados, acotandonos solo a los relevantes
para la consigna, aplicando de este modo la busqueda del m\'inimo a travez de backtracking.\\

%Previa explicaci\'on de como fue realizada la implentaci\'on explicaremos que ser\'ia para nosotros un $parValido$.

Para implementar la solución se atomizo cada ciclo a 1 solo viaje: el env\'io de gente de A a B o el regreso de \textit{'faroleros'} de B a A. De esta forma, el backtracking se puede realizar en 
la desición de la gente que va de A a B, independientemente de la elección de los faroleros.

Dadas las restricciones del problema, se pudieron aplicar las siguientes podas sobre el arbol de posibilidades:
\begin{itemize}
	\item Las combinaciones evaluadas son, en el caso de enviar 2 personas, a duplas sin repeticiones.
	\item Los viajes, tanto de paso de A a B como de B a A que generan un 'desbalance', es decir que en alg\'un lado hay más canibales que arqueologos, son obviados.
	\item Las secuencias de viajes que tomen más tiempo que una previamente calculada se descartan.
	\item Solo se consideran viajes de A a B en duplas y de una sola persona de B a A.
\end{itemize}

En base a la evaluación de todas las soluciones encontradas por el algoritmo, nos quedamos con aquella que menor tiempo acumule con los viajes.