La elección de las podas tomadas en el backtracking se debieron a las siguientes razones:

Siendo el conjunto total de combinaciones existentes contenedor de tuplas de elementos repetidos, (por ejemplo, para las personas P1 y P2, existe la tupla <P1, P2> y <P2, P1>), es trivial ver que evaluar estos casos carece de sentido, ya que es exactamente el mismo resultado el que se obtendra con una que con otra.\\

La consigna del ejercicio plantea la necesidad de mantener un balance entre los personajes involucrados. En una aproximación con fuerza bruta, se desestimarían las soluciones que conlleven algun desbalance en la sucesion de traslados de los personajes, para lo cual primero se deberían calcular todas las posibilidades y luego analizarlas una por una, recorriendo cada uno de sus pasos, buscando algun desbalance. Para evitar tener que analizar, al final de la construccion de las soluciones, si son o no válidas, se decide chequear en cada paso de la construccion de cada solución, si el mismo produce una instancia balaceada. Al podar de esa forma, solo nos quedaremos al final del algoritmo con las soluciones que son validas.\\

Teniendo ya las soluciones luego de ejecutar el algoritmo, se debería encontrar aquella que produzca el mínimo tiempo de viaje de los personajes. Esto quiere decir que habrá soluciones que estemos desestimando que consumieron tiempo en ser construidas. Lo que se resuelve es, al calcular una solucion factible, se guarda el tiempo logrado, y todas las siguientes soluciones, si al construirse sobrepasan este tiempo, entonces quedan automáticamente desestimadas.\\

Por ultimo se analizan las posibilidades que se tiene al efectuar un paso. Para esto se considera al envio de personas y al retorno del farol como un PASO, notando \textbf{n-m} al envio de n personas al lado "B" y al retorno de m personas al lado "A", dando lugar a las siguientes posibilidades: {1-1, 1-2, 2-1, 2-2}.

Notemos que para hallar una solución, es necesario que, en algun punto, la cantidad de personas en el lado "A" decremente, y se incremente en el lado "B", de forma tal que terminen todas del lado "B"; en caso contrario la secuencia de pasos nunca terminará.\\ Observando las opciones disponibles, se observa que, por paso, puede decrementarse la cantidad de personas del lado "A" en 1 como máximo, con lo cual,en la sucesion de pasos que representa una solución, hay una subsucesion en la que cada paso decrementa en 1 la cantidad de personas en "A".\\ Analizando los casos posibles, es evidente ver que el único paso que puede construir esta subsucesión es el 2-1, ya que los demás mantienen o incrementan la cantidad en "A".\\\\

Podemos ver, entonces, que de poder construir una sucesión plenamente con pasos 2-1 (pasos efectivos), ésta será minima en tiempo empleado. Utilizar backtracking solo con el 2-1 (backtracking 2-1), entre 2 pasos consecutivos, se evalúan todas las combinaciones de 2 conjuntos con diferencia en 1 en su cantidad de elementos. Resta ver que al sacar las demas posibilidades no se pierden soluciones:

De utilizar el paso 2-1 conjunto al 1-1, la distancia entre pasos efectivos se puede incremetar. Como al utilizar el backtracking 2-1 se puede obtener una transicion entre estos 2 pasos de forma más directa, el agregar el 1-1 solo incrementa el tiempo de la solución,con lo cual puede podarse.\\ 

De utilizar el 2-1 con el 2-2, se produce el mismo efecto que en el caso anterior.\\

De utilizar el 2-1 con el 1-2, se estarìa volviendo a un paso anterior en cantidad de personas de un lado y otro, el cual ya habría sido evaluado en el caso del backtracking 2-1.

 

 