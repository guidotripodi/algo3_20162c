
En nuestro algoritmo como hemos mencionado anteriormente en la explicaci\'on, la etapa m\'as importante es a la hora de obtener las pesas, la segunda, en donde realizamos un ciclo que va disminuyendo el valor $equilibrioActual$ hasta llegar a 0. 

Queremos demostrar que en cada iteraci\'on del ciclo es posible disminuir dicho valor hasta llegar a 0 siempre y cuando, la diferencia que nombramos $N$ es menor que $equilibrioActual$:\\

%Para facilitar esta demostraci\'on nos abstraemos del m\'odulo utilizado para N ya que tomar m\'odulo a $equilibrioActual - {3^i} = N < 0$ es lo mismo que multiplicar $N$ o sea el resultado de la diferencia por -1 e iterear nuevamente.\\ 

Llamemos $j$ a cada una de las iteraciones del ciclo la cual inicializaremos $j = i$.\\
Por lo tanto para probar que dentro del ciclo se disminuye hasta llegar a $equilibrioActual$ = 0 probemos que:\\

Para $j$, luego de la iteraci\'on $j-esima$ el valor $N$ es $< equilibrioActual$ hasta llegar a $N = 0$ o $N = 1$. Por lo tanto como iniciamos con un valor i donde ${3^i}\geq equilibrioActual$ como $j=i$ si ${3^j} = equilibrioActual$. Llamemos $equilibrioActual' = equilibrioActual$ como valor inicial. Realizando la primera iteraci\'on j $equilibrioActual' - {3^j} = N$, $N < equilibrioActual'$ por lo cual para la primera iteraci\'on j $N < equilibrioActual' \leq equilibrioActual$, lo cual si no fuese cierto ${3^j}$ no ser\'ia igual a equilibrioActual, contradiciendo la igualdad de los mismos enunciada anteriormente.\\

Luego, si ${3^j} > equilibrioActual$ con $j=i$. $equilibrioActual - {3^j} = |N|$, Si $|N| > equilibrioActual$ ser\'a el caso no utilizado como mencionamos anteriormente, por lo cual pasamos a la pr\'oxima iteraci\'on $j'= j-1$ Siempre y cuando $N \neq 0 \wedge N \neq 1$ (Estos casos finalizan nuestra ejecuci\'on de ciclo). Si $|N| < equilibrioActual$, llamamos $equilibrioActual' = |N|$, por definici\'on $equilibrioActual' < equilibrioActual$ con lo cual, habremos disminuido para esta $j-esima$ iteraci\'on el valor de $equilibrioActual$, en caso de que $N = 0$ habremos finalizado el ciclo habiendo alcanzado el valor de equilibrio, lo mismo suceder\'a para el caso de que $N = 1$ ya que contaremos con la pesa ${3^0}$ que no utilizamos nunca ya que arrancamos siempre por la potencia m\'as grande.\\

Siguiendo con los casos, si ${3^j} < equilibrioActual$, $equilibrioActual - {3^j} = |N|$ $N$ siempre ser\'a menor que $equilibrioActual$ por lo tanto para este caso en cada iteraci\'on j-esima que cumpla ${3^j} < equilibrioActual$, y $equilibrioActual' = N$, $equilibrioActual' < equilibrioActual$. Adem\'as, si el resultado de la diferencia es 0 o 1, significar\'a que hemos alcanzado el equilibrio que necesitabamos por lo cual finaliza el ciclo.\\


Luego, como guardamos en nuestro array todas las pesas con signo positivo si fueron restando o con signo negativo si terminaron sumando, nos queda que:\\

\[
\sum_{x=1}^{arrayPesasUtilizadas.size}arrayPesasUtilizadas{x}= P 
\]

