\vspace*{1em}

Realizando pasos 2-1, cada solución construída tendrá {\bf a lo sumo n-1 pasos}, siendo n la cantidad total de personas en la comitiva.

En el primer paso se analizan $\binom {n}{2}$ parejas posibles a enviar y por cada una de ellas se eligirá de entre 2 personas para ser el farlero.\\

En el segundo paso se tendran $\binom {n-1}{2}$ parejas posibles y 3 faroleros para elegir.\\

Para el (i)-esimo paso se tendrán $\binom {n-(i-1)}{2}$ parejas e i+1 faroleros.\\

Por cada una de las posibilidades de cada paso, se deberan analizar la cantidad total de
posibilidades que haya en el paso subsiguiente, para dar lugar asi al analisis de todas las combinaciones posibles, es decir: siendo que el segundo paso se ejecuta $\binom {n}{2}$ veces, el tercero $\binom {n}{2} \binom{n-1} {2}.3$, podemos ver que en combinación, todos los pasos demandarían:
\[
\prod_{i=1}^{n-1}\binom {n-(i-1)}{2}*(i+1) = \prod_{i=1}^{n-1}\binom {n-(i-1)}{2} * \prod_{i=0}^{n-1}(i+1)
\]

\[
n!\prod_{i=0}^{n-2}\binom {n-i}{2} = n!\prod_{i=0}^{n-2}\frac{(n-i)(n-(i-1))}{2}
\]

\[
\frac{ n!}{2^{n-2}}.\prod_{i=0}^{n-2}(n-i).\prod_{i=0}^{n-2}(n-(i-1)) = \frac{n!}{2^{n-2}}.n!.\frac{(n+1)!}{2}\leq \frac{n!^{2}.(n+1)!}{2^{n-1}}\in O(n.n!^{3})
\]