
ACA VA EL PSEUDOCODIGO
LA IDEA DE USAR ESTA MACRO ESTA BUENA PORQUE YA PODEMOS IR PONIENDO LA COMPLEJIDAD
Y DSP CUANDO NOS TOQUE HACER LA DEMOSTRACION SALE MAS FACIL

\begin{algorithm}[H]
\caption{CRUZANDO EL PUENTE}
\begin{algorithmic}[1]
\Function{minimoTiepoDeCruce}{\param{in}{cableSize}{Integer}, \param{in}{stationDistances}{List<Integer>}}{$\ensuremath{\rightarrow}$ \param{out}{res}{Integer}}
\state List<Integer> distanceDifferences $\gets$ List<Integer>(vacio) \hfill //O(1)
\state Integer lastStation $\gets$ 0 \hfill //O(1)
\state Integer distance $\gets$ 0 \hfill //O(1)
\while{i < stationDistances} \hfill //O(N)
\state distance $\gets$ stationDistances \hfill //O(1)
\state distanceDifferences.Agregar(distance - lastStation) \hfill //O(1)
\state lastStation $\gets$ distance \hfill //O(1)
\state i++ \hfill //O(1)
\endwhile
\state res $\gets$ getMaxRangeLength(cableSize, distanceDifferences) \hfill //O(N)
\EndFunction 
\end{algorithmic}
\hrule
\complejidad{n}
\end{algorithm}

