\begin{algorithm}[H]
\caption{CRUZANDO EL PUENTE}
\begin{algorithmic}[1]
\Function{EJ1}{\param{in}{ }{Integer}, \param{in}{ }{List<Integer>}}{$\ensuremath{\rightarrow}$ \param{out}{res}{Integer}}
\state creo bool exitoBackPar con valor verdadero \hfill //O(1)
\state creo bool exitoBackLampara con valor verdadero \hfill //O(1)
\while{exitoBackLampara $\vee$ exitoBackPar} \hfill //O($n!^{3} \ast n$)
\If{$\exists$ parPosible()} \hfill //O(1)
\state par $\gets$ dameParPosible() \hfill //O(1)
\Else
\state exitoBackLampara $\gets$ backtrackRetorno(farolero) \hfill //O(1)
\endif
\If{$\exists$ faroleroPosible()} \hfill //O(1)
\state retornarLampara(farolero) \hfill //O(1)
\Else
\state exitoBackPar $\gets$ backtrackPar(par) \hfill //O(1)
\endif
\If{pasaronTodos()} \hfill //O(1)
\state guardarTiempo() \hfill //O(1)
\endif
\endwhile
\EndFunction 
\end{algorithmic}
\hrule
\complejidad{$n!^{3} \ast n$}
\end{algorithm}

NOTA: LAS COMPLEJIDADES DE LAS FUNCIONES INVOLUCRADAS NO ES O(1). 
HAY QUE PONER LAS CORRECTAS Y DESARROLLAR UNA LISTA DE PUNTOS INDICANDO LOS PASOS QUE REALIZA A GRANDES RASGOS Y COMO CADA UNO CUMPE CON UNA COMPLEJIDAD QUE EN TOTAL ES LA QUE SE INDICA

