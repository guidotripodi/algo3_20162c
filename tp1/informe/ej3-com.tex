El algoritmo por cada uno de los N elementos construye una matriz de $K_{1} \times K_{2} \times...\times K{m}$, y las guarda para poder recuperar al final, los elementos involucrados en la solución.

Para construir estas matrices se recorre cada uno de sus $K_{1} \times K_{2} \times...\times K_{m}$ indices para calcular el valor correspondiente en tiempo constante; habiendo n elementos, la construccion de las matrices se realiza en O($N \ast \prod_{i=1}^{m}K_{i}$). Con lo cual podemos ver lo siguiente: dado que la suma de la cantidad de cada tipo de elemento es la cantidad de elementos en si:

\[
N \ast \prod_{i=1}^{m}K_{i} = \sum C_{i} \ast \prod_{i=1}^{m}K_{i}
\]

Si tomamos la cota de 3 mochilas dada por la consigna, y acotamos las capacidades de las mochila por la de aquella de mayor capacidad (capacidad K) se tiene:

\[
\sum C_{i} \ast \prod_{i=1}^{3} K_{i} \leq
N \ast \prod_{i=1}^{3}K_{i} = \sum C_{i} \ast \prod_{i=1}^{3}K = \sum C_{i} \ast  K^{3}
\]  

Podemos concluir que el algoritmo respeta la cota requerida:


\[
\sum C_{i} \ast  K^{3} \in O( \sum C_{i} \ast (\sum K_{i})^3)
\]