
Nuestro algoritmo como mencionamos anteriormente presenta 3 etapas.\\
La primera de ellas consta en recorrer desde $3^0$ hasta $3^i$ donde este ultimo sea igual a $P$ o en su defecto el inmediato mayor. Por lo tanto mostraremos que recorrer hasta un i donde $3^i$ sea igual o inmediatamente mayor a P es menor o igual a $\sqrt{P}$.\\


Si $i = 0$ $\Rightarrow$ terminamos.\\
Luego sea $3^i \geq P \geq 3^{i-1}$ con $i > 0$. Queremos ver que $i \leq \sqrt{P}$:\\
Se que por definici\'on $P \geq 3^{i-1}$ $\Rightarrow$ $\sqrt{P} \geq \sqrt{3^{i-1}}$\\
Veamos que $\sqrt{3^{i-1}} \geq i \Rightarrow 3^{i-1} \geq {i^2}$. Para $i = 1$ tenemos que $3^{1-1} \geq 1$ siempre. Luego, para $i > 1$ como ${3^{i-1}}$ es creciente y mayor o igual que $i^2$ se cumple siempre esta desigualdad. Por lo tanto queda probado que recorrer hasta un i tal que $3^i \geq P$ se encuentra en el orden de  O($\sqrt{P}$).\\


 Luego, creamos dos enteros \textit{saldoEnBalanza} y \textit{N} y un array \textit{arrayPesasUtilizadas} inicializado vacio, por lo tanto, como son enteros y el array es vacio esto insume O(1), finalizando as\'i la primera etapa.\\
Luego, la segunda etapa y m\'as importante de nuestro algoritmo, consiste en un ciclo donde realizaremos iterativamente la busqueda de las pesas que, sumando sus valores, nos de el valor $P$. Dicho ciclo en el peor de los casos recorrer\'a desde el valor de i que conten\'ia la pesa de mayor valor hasta i = 0, lo cual ser\'a O($\sqrt{P}$) ya que al trabajar con potencias de 3 la cantidad de vueltas del ciclo entraran en el orden de $\sqrt{P}$ como hab\'iamos demostrado anteriormente.\\ 

Dentro de este ciclo, realizamos 6 comparaciones en O(1) las cuales son:

\begin{itemize}
\item si N=0, si N=1, aqu\'i agregamos el valor de la pesa al array y terminamos el ciclo, esto insumir\'a O(1)
 \item si N es menor al valor de \textit{saldoEnBalanza}, estas opcion no finaliza el ciclo,
 agrega la pesa n el array, modifica el valor de saldoEnBalanza por el valor de N y disminuye en 1 a i, luego se chequea aqui mismo si $N < 0$ de ser verdadero se modifica el valor de $estaEnNegativo$ por verdadero o por falso en caso de ser falsa la guarda, lo cual insumir\'a O(1) cada una de las operaciones mencionadas, luego se chequea aqui mismo si $N < 0$ de ser verdadero se modifica el valor de $estaEnNegativo$ por verdadero o por falso en caso de ser falsa la guarda
 \item por ultimo, si $N \geq saldoEnBalanza$ solamente descontamos en uno a i para continuar con el ciclo.
\end{itemize} 
Una vez finalizado esto y por consiguiente la segunda etapa, pasamos a la tercera la cual consiste en guardar en \textit{arrayI} o \textit{arrayD} los valores de los elementos del \textit{arrayPesasUtilizadas} para colocarlos en la balanza del lado derecho o del izquierdo.\\
Para realizar esto recorremos el array \textit{arrayPesasUtilizadas} el cual en el peor de los casos tendr\'a todas las pesas, lo cual como vimoss recorrer la totalidad de elementos del array se encuentra en el orden de O($\sqrt{P}$).\\

Luego de ver esto, dentro del ciclo realizamos operaciones en O(1).\\
Por ultimo, devolvemos los array invirtiendo las posiciones de los elementos lo que insumir\'a en el peor de los caso O($\sqrt{P}$).\\

En conclusi\'on nuestro algoritmo realiza 4 ciclos que demandan en el peor de los casos O($\sqrt{P}$) donde dentro de los mismos se realizan operaciones en O(1), por lo tanto nuestra complejidad final ser\'a
O($\sqrt{P}$).




