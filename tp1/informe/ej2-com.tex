
Nuestro algoritmo como mencionamos anteriormente presenta 3 ciclos predominantes de los cuales uno corresponde a la busqueda binaria.\\
El primero de ellos consta en recorrer desde $3^0$ hasta $3^i$ donde la suma de estos sea igual a $P$ o en su defecto el inmediato mayor. Por lo tanto, como la suma se realiza en O(1), mostraremos que recorrer hasta un i donde la suma de dichos valores sea igual o inmediatamente mayor a P es menor o igual a $\sqrt{P}$.\\

Si $i = 0$ $\Rightarrow$ terminamos.\\
Luego sea $3^i \geq P \geq 3^{i-1}$ con $i > 0$. Queremos ver que $i \leq \sqrt{P}$:\\
Sabemos que $P \geq 3^{i-1}$ $\Rightarrow$ $\sqrt{P} \geq \sqrt{3^{i-1}}$\\
Veamos que $\sqrt{3^{i-1}} \geq i \Rightarrow 3^{i-1} \geq {i^2}$. Para $i = 1$ tenemos que $3^{1-1} \geq 1$ siempre. Luego, para $i > 1$ como ${3^{i-1}}$ es creciente y mayor o igual que $i^2$ se cumple siempre esta desigualdad. Por lo tanto queda probado que recorrer hasta un i tal que \[
\sum_{x=0}^{i}3^{i} \geq P \] se encuentra en el orden de  O($\sqrt{P}$).\\

 Luego, creamos un long long \textit{size} inicializado en i+1 y un array \textit{sumasParciales} de tamaño size inicializado vacio, por lo tanto, la creaci\'on de la variable size y el array vacio insumir\'an O(1) y O($\sqrt{P}$).\\
 
Siguiendo el desarrollo del algoritmo, pasamos a nuestro segundo ciclo, en el cual llenaremos el array $sumasParciales$, como vimos anteriormente iterar desde el valor i hasta 0 es O($\sqrt{P}$), y dentro de dicho ciclo lo \'unico que hacemos es ir guardando en la posici\'on i-esima del array el valor de sumaParcial - 3$^{i-1}$ y a sumaParcial le guardamos el valor de sumaParcial - 3$^{i-1}$. Como estas dos operaciones se realizan en O(1), nuestro segundo ciclo terminar\'a insumiendo O($\sqrt{P}$).\\

Luego, nuestro tercer y \'ultimo ciclo, corresponde a la b\'usqueda binaria, la cual se realiz\'a en O(lg($\sqrt{P}$)) como en toda busqueda binaria, trabajaremos con nuestro array $sumasParciales$ chequeando si en la mitad del array nuestra sumaParcial es mayor o igual al valor en m\'odulo de $equilibrioActual$. En caso de que fuese verdadero, chequeamos si el valor de $equilibrioActual$ es mayor o menor a 0. Si es menor a 0, sumaremos nuestra pesa correspondiente al \'indice en el que estamos de nuestro array $sumasParciales$, al valor de $equilibrioActual$ y guardaremos nuestra pesa en el $arrayD$ que simboliza al plato derecho de la balanza. Si es mayor a 0, en vez de sumarla la restamos y la guardamos en el array $arrayI$ que simboliza el otro plato. Siguiendo el razonamiento de la busqueda binaria, volveremos a partir nuestro array en dos y haremos el mismo chequeo.\\
En caso de que el valor en m\'odulo de $equilibrioActual$ sea mayor que la mitad de $sumasParciales$ nos quedaremos con la mitad m\'as grande del arreglo e iteraremos nuevamente.\\
Una vez llegado a 0 el valor de $equilibrioActual$ saldremos del ciclo. Como describimos dentro del ciclo realizaremos sumas, restas y chequeos los cuales se realizaran todos en O(1), por lo tanto. Nuestro tercer ciclo insumir\'a O(lg($\sqrt{P}$)).\\

Fuera de este \'ultimo ciclo, tendremos nuestras pesas ordenadas de mayor a menor en $arrayD$ y en $arrayI$, bastar\'a con invertir los arreglos para que queden de menor a mayor y devolver los mismos, finalizando as\'i nuestro algoritmo. Dicho invertir costar\'a O($\#$elementosArrayD) y O($\#$elementosArrayI), que como demostramos anteriormente en el caso de que todas las pesas fueran a parar a un \'unico plato y naturalmente a un \'unico array O($\#$elementos) $\leq$ O($\sqrt{P}$).\\

Por lo tanto, nuestro algoritmo realizar\'a en su defecto 3 ciclos (como vimos, se puede dar el caso de invertir el array y que est\'en todas las pesas en un \'unico array) O($\sqrt{P}$) y el ciclo de la busqueda binaria O(lg($\sqrt{P}$)), nos queda que la complejidad total de nuestro algoritmo es O($\sqrt{P}$).\\

\textbf{Complejidad total:} O($\sqrt{P}$) [O(1) + O(1)] + O(1) + O($\sqrt{P}$) + O($\sqrt{P}$) [O(1) + O(1)] + O(lg($\sqrt{P}$))[O(1) + O(1) + O(1) + O(1) + O(1) + O(1)] = \textbf{O($\sqrt{P}$)}