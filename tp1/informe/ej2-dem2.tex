En nuestro algoritmo como hemos mencionado anteriormente en la explicaci\'on del mismo, la etapa m\'as importante es a la hora de obtener las pesas que equilibran la balanza, la segunda, en donde realizamos un ciclo que va disminuyendo el valor $equilibrioActual$ hasta llegar a 0. 

Este algoritmo utiliza una propiedad particular que es la de poder generar todos los números entre 0 y $\sum_{i=0}^{n}(3^i)$ con potencias de 3 diferentes (El 0 se incluye por definición). Pero para que esto sea válido debemos demostrarlo.\\

Veamos para empezar que podemos generar todos los números entre $[0, 1]$ que son 0 y 1.\\

Este es un caso bastante trivial, por lo tanto veamos que sucede en el intervalo 

\begin{equation}
[0,  \sum_{i=0}^{1}(3^i)] = [0, 4]
\end{equation}

\begin{itemize}
\item 4 = 3+1
\item 2 = 3-1
\end{itemize}

Parece pues que para el caso base, es decir el primer intervalo, podemos generarlos todos con potencias de 3 diferentes. 

Asumamos pues que esto vale para $i = n \in \mathbb{N}$ y demostremos que vale para $n+1$ \\

Es decir, puedo generar todos los números en el intervalo 

\begin{equation}
[0, \sum_{i=0}^{n}(3^i)]
\end{equation} \label{eq:hi}

Veamos que podemos lograr lo mismo para 

\begin{equation}
[0,  \sum_{i=0}^{n+1}(3^i)]
\end{equation}\label{eq:pi}

Pues veamos que 

\begin{equation}
[0, \sum_{i=0}^{n+1}(3^i)] = [0, \sum_{i=0}^{n}(3^i)] + 3^{n+1} 
\end{equation} 

Con lo cual ya puede verse que los numeros entre 0 y $\sum_{i=0}^{n}(3^i)$ podemos generarlos por hipótesis inductiva \eqref{eq:hi}.\\

Luego notemos que 

\begin{equation}
3^{n+1} = 3 \ast 3^n. 
\end{equation}

Como $3^n$ $<$ $\sum_{i=0}^{n}(3^i)$, $3^n$ está dentro del intervalo de la hipótesis inductiva \eqref{eq:hi} así que tambien podemos formarlo con potencias de 3, y en particular cualquier número menor a $3^n$ usando la misma hipótesis. \\

Luego podemos generar cualquier $x \in \mathbb{N}$, $x \leq$ $\sum_{i=0}^{n+1}(3^i)$. \\

Por lo tanto queda probado que la hipótesis inductiva vale $\forall n \in \mathbb{N}$ \\

Además notemos que el número más cercano a $x$ será la $maxima$ $potencia$ $de$ $3$ $en$ $[0, $ $\sum_{i=0}^{n+1}(3^i)]$ es decir $3^{n+1}$.
 
Pues este es el intervalo que genera a $x$ y sabemos que $\sum_{i=0}^{n}(3^i)$ $<$ $3^{n+1}$ y que
$\sum_{i=0}^{n}(3^i)$ $<$ $x$.\\

Y como al restar $x$ por $3^{n+1}$ obtenemos un número más pequeño que $\sum_{i=0}^{n}(3^i)$ en módulo, pues (el caso negativo es simétrico)

\begin{equation}
x < \sum_{i=0}^{n+1}(3^i) =
x < \sum_{i=0}^{n}(3^i) + 3^{n+1} =
x - 3^{n+1} < \sum_{i=0}^{n}(3^i) 
\end{equation}

Entonces nunca se repetiran potencias de 3 en el proceso.

Con esto se concluye que se pueden generar todos los números mediante potencias de 3 únicas.