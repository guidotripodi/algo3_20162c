\indent Para verificar el correcto funcionamiento de nuestro algoritmo , elaboramos disversos tests,
los cuales ser\'an enunciados a continuaci\'on.\\

\begin{center}
 \textbf{Todos los arqueologos y canibales presentan la misma velocidad}
\end{center}
 Para este tipo de testeo mostraremos a continuaci\'on un ejemplo del mismo, exponiendo su respectivo resultado.\\

 Con un:\\\\
  \indent  
	
	$Cantidad de arqueologos: $  
	$Cantidad de canibales: $  
	$Velocidad de arqueologos: $  
	$Velocidad de canibales: $  \\
  
  Obtuvimos el siguiente resultado:\\
  
  $Velocidad de cruce total: $

 \begin{center}
 \textbf{No hay canibales}
\end{center}

Para este tipo de testeo mostraremos a continuaci\'on un ejemplo del mismo, exponiendo su respectivo resultado.\\

 Con un:\\
  \indent 	
  	$Cantidad de arqueologos: $  
	$Cantidad de canibales: $  
	$Velocidad de arqueologos: $  
	$Velocidad de canibales: $  \\
  
  Obtuvimos el siguiente resultado:\\
  
  $Velocidad de cruce total: $\\

\begin{center}
 \textbf{Estaciones con bastante distancia intercaladas con estaciones m\'as cercanas}
\end{center}

Aqu\'i veremos, un ejemplo del conjunto de test de este tipo, exponiendo su respectivo resultado.\\

\indent $Rollo$ $de$ $Cable:$ 200\\
$Lista de estaciones:$ 15 16 17  40 41 42 70 71 72 100 101 102 140 141 142 170 171 172 200 201 202 230 231 232\\

Obtuvimos el siguiente resultado:\\ 
  
  $Cantidad$ $de$ $estaciones$ $conectadas:$ 21
