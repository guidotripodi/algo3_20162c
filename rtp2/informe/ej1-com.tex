 El algoritmo BFS original, tiene una complejidad de O($|V|+|E|$), siendo V el conjunto de nodos y $E$ el de aristas del grafo que se le pasa como parámetro. El sumando $|V|$ es obtenido a partir de que se encolan necesariamente cada uno de los $|V|$ nodos {\bf 1 sola vez}. Por otro lado, como cada nodo chequea 1 vez cada una de sus adyacencias (para ver si las encola o no), entonces cada nodo n es chequeado $d(n)$ veces (es decir hay O(|E|) chequeos en total). \\

Siendo la entrada una matriz de $F \times C = N$ (donde F es la cantidad de filas y C la cantidad de columnas), cada celda es transformada en un nodo, con lo cual el grafo resultante tendrá N nodos. Cada baldosa puede recorrerse a lo sumo en 2 direcciones y 4 sentidos distintos, es decir, los nodos que las representan tienen grado menor o igual a 4, osea orden constante. De aplicar BFS sobre el grafo de nuestro problema, la complejidad sería O($N + 4N$) $\subseteq$ O($N$).
No obstante, al modificar el BFS permitiendo que se puedan "visitar nodos ya visitados", la complejidad se ve afectada, ya que en la cola que guarda los nodos a evaluar permite reencolar a cada uno varias veces (permitiendo su reevaluaci\\on).\\

La cantidad de veces en la que es posible reencolar cada nodo, esta dada por la cantidad de caminos que pasen por él, los cuales comparten la caracteristica de que cada uno de ellos mejora la cantidad de paredes a romper necesarias para alcanzarlo. Como ningún camino puede derribar más de "P" paredes y bajo la l\'ogica de que un camino podrá revisitar a un nodo (por lo tanto reencolarlo) si lo alcanza con una menor cantidad de paredes derribadas que el camino anterior, entonces solo P caminos podrán mejorar la condición de acceso al nodo. Por lo tanto la cantidad de veces que se encolarán nodos sera O($N \times P$) determinando ,de esta forma, la compliejidad del algoritmo propuesto.



%Como existe la posibilidad de mejorar nuestro camino hasta cada uno de los nodos, rompiendo una cantidad puntual de paredes, podremos recorrer P veces cada nodo para chequear si existe un camino de menor tiempo. Se podr\'a recorrer hasta P veces ya que es la cantidad m\'axima de paredes posibles a romper, ya que chequearemos a cada nodo.

%Cada nodo se podr\'a volver a recorrer si y solo si es posible llegar a él por un camino que rompe menos paredes. Dado que solo se pueden romper hasta P paredes, cada camino a cada nodo se podr\'a mejorar a lo sumo P veces.


%es posible mejorar el camino obtenido hasta el momento rompiendo una cantidad puntual de paredes la cual deber\'a ser menor a P
