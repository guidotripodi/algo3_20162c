Siendo $F$ la cantidad de filas y $C$ la cantidad de columnas de la entrada, $F \ast C$ es la cantidad de puntos caminables máximos que puede tener una instancia. El grafo impl\'icito tiene entonces $V(G) = F \ast C $ nodos.

Dado que la implementación del algoritmo de Kruskal utiliza la técnica de union-find con tiempo de uni\'on y b\'usqueda amortizados $E \ast log(E)$ siendo $E$ la cantidad de aristas del grafo y  ordenamiento inicial de las aristas que se logra tambi\'en en $E \ast log(E)$ la complejidad total será de $O(E \ast log(E))$.

Dado que el grafo más completo que puede ofrecer este problema es cuando se tiene una sola sala sin paredes, la cantidad de aristas total será 4$\ast  V(G)$ = 4$\ast  F \ast C$ dando como resultado un orden de complejidad 4 $\ast F \ast C \ast log(4 \ast F \ast C)$ $\subseteq$ $O(F \ast C \ast log(F \ast C))$.
