\subsubsection*{Ejercicio 1}
\begin{itemize}
\item Se realiz\'o una explicaci\'on de la implementaci\'on del algoritmo adicional al pseudoc\'odigo
\item Se quitaron las podas al c\'odigo implementado y se realizo una nueva medici\'on de tiempos con la respectiva comparaci\'on del algoritmo con y sin podas.
\end{itemize}

\subsubsection*{Ejercicio 2}
\begin{itemize}
\item Se realiz\'o una explicaci\'on de la implementaci\'on del algoritmo adicional al pseudoc\'odigo
\item Se creó un nuevo conjunto de instancias random y se introdujo la familia \textbf{grupos de gimnasios}
\item Se realizaron nuevas experimentaciones con un total de 470 elementos (pokeparadas + gimnasios) obteniendo nuevas conclusiones
\end{itemize}

\subsubsection*{Ejercicio 3}
\begin{itemize}
\item Se realiz\'o una explicaci\'on de la implementaci\'on del algoritmo adicional al pseudoc\'odigo
\item Se realizaron nuevas experimentaciones con un total de 470 elementos obteniendo nuevas conclusiones
\end{itemize}

\subsubsection*{Ejercicio 4}
\begin{itemize}
\item Se realiz\'o una explicaci\'on de la implementaci\'on del algoritmo adicional al pseudoc\'odigo
\item Se realizaron nuevas experimentaciones con un total de 470 elementos obteniendo nuevas conclusiones
\end{itemize}

\subsubsection*{Ejercicio 5}
\begin{itemize}
\item Se creó un nuevo conjunto de instancias random y la familia \textbf{grupos de gimnasios}
\item Se realizaron nuevas experimentaciones con un total de 470 elementos obteniendo nuevas conclusiones
\end{itemize}