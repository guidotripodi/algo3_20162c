Como comentario inicial, se presenta el algoritmo $Tabu$ $Search$ con ambos criterios de parada en el ciclo principal (cantidad de iteraciones límite y cantidad de iteraciones sin mejora) ya que ambos pueden aplicarse a la vez o por separado que es como serán estudiados en la experimentación.

El algoritmo itera sobre una solución inicial $S_o$ que es un vector con los $id$'s de las pokeparadas y gimnasios como fue definido en el ejericio tres. \ref{sec:alg3} 
Se utiliza el vector $solucionActual$ para guardar el recorrido que se utiliza para recorrer vecindades. Mientras que $mejorSolucion$ guarda el mejor recorrido encontrado en todas las vecindades analizadas y será este el encargado de hacer valer la condición de corte de cantidad de iteraciones sin mejora.

El algoritmo procede a calcular la vecindad de $solucionActual$ filtrando aquellos recorridos no válidos para el problema en cuestión (las condiciones de validez de una solución fueron dadas en el ejercicio dos) y luego busca el mejor candidato dentro de esta vecindad, que será aquella con menor costo y que no sea un recorrido marcado como tabú (es decir que previamente ya fue tomado).

Si no obtuvo un candidato, procede a utilizar la función de aspiración, que será elegir aquel recorrido en el vecindario que sea menos tabú.

En cualquier caso, siempre se obtienen las aristas modificadas para llegar a este recorrido dentro de la vecindad (esto depende de la búsqueda local utilizada) y se agregan las mismas a la lista tabú.

Una vez hecho esto se actualiza el $mejorCosto$ y la $mejorSolucion$ si el recorrido encontrado mejora el $mejorCosto$ actual. Si esto no sucede, se aumenta en una unidad la cantidad de iteraciones sin mejora. 

El algoritmo finaliza cuando se alcanza la cantidad de iteraciones límite o la cantidad de iteraciones sin mejora límite. Una u otra dependen de la elección del usuario.

\begin{algorithm}[H] %or another one check
 \Fn{tabuSearch()}{
 %     \SetAlgoLined
 $S_o$ es la solución provista por el algoritmo greedy \\
 ConjuntoTabu attributosTabu \hfill O(n) \\ 
 Recorrido mejorSolucion $\leftarrow$ $S_o$ \hfill O($n$)\\
 Recorrido  solucionActual $\leftarrow$ $S_o$ \hfill O($n$)\\ 
 entero mejorCosto $\leftarrow$ calcularCosto(mejorSolucion) \hfill O($n$)\\
entero iter = 0\\
entero MaxIter = cantidad pokeparadas+cantidad gimnasios \\
entero TenorTabu = MaxIter \\ 
entero MaxNoMejora = 4 \\ 

 \For{iter < MaxIter $\wedge$ entero noMejora < maxNoMejora }{
	 \hfill ciclo: O($Critero Parada$)\\
	Recorrido mejorCandidato \hfill O(1) \\
  	Lista<Arista> aristasModificadas \hfill O(1)\\
  	entero costoMejorCandidato $\leftarrow$ -1 \hfill O(1)\\
  	
	Conjunto<Recorrido, Lista<Arista> > vecindadActual $\leftarrow$ vecindadFiltrada(solucionActual) \hfill O($n^4$)\\
  	
  	\For{par en vecindadActual}{
		\hfill Ciclo: O($n^4$)\\
  		Recorrido candidatoActual $\leftarrow$ par.first \hfill O(n)\\
  		entero costoActual $\leftarrow$ calcularCosto(candidatoActual) \hfill O($n$)\\
		costoMejorCandidato  	$\leftarrow$ calcularCosto(mejorCandidatol) \hfill O($n$)\\
		
		\If{costoActual < costoMejor $\vee$ (!tabuCount(atributosTabu, candidatoActual) $\wedge$ (costoActual < costoMejorCandidato $\vee$ costoMejorVecino = -1)}{
			\hfill Condicion: O($n$)\\
			aristasModificadas $\leftarrow$ par.second \hfill O(1) \\%|par.second|<=4
			mejorCandidato $\leftarrow$ candidatoActual \hfill O($n$)\\
		}
  	}  	
	\hfill Total del ciclo: O($n^5$)

	\If{no se encontro mejorCandidato}{
		\hfill Condicion: O(1)\\
		<Recorrido, Lista<Arista> > menosTabu $\leftarrow$ funcionAspiracion(atributosTabu, vecindadActual) 
		\hfill O($n^5*log(TenorTabu)$)\\%
		
		mejorCandidato $\leftarrow$ menosTabu.first \hfill O($n$)\\
		aristasModificadas $\leftarrow$ menosTabu.second \hfill O(1)\\
	}  	
	
	costoMejorCandidato = calcularCosto(mejorCandidato); \hfill O($n$) \\
	
	solucionActual $\leftarrow$ mejorCandidato \hfill O($n$)\\
		
	\If{costoMejorCandidato $<$ mejorCosto}{
		\hfill Condicion: O(1)\\
		mejorSolucion $\leftarrow$ mejorCandidato \hfill O($n$)\\
		mejorCosto $\leftarrow$ costoMejorCandidato \hfill O(1)\\
		noMejora=0\\
	}\Else{
		noMejora+=1\\
	}
		iter+=1 \\
}
agregarAtributosTabu(AristasModificadas)\\
devolver mejorSolucion \\
}

\end{algorithm}
	
\begin{algorithm}[H]

\Fn{agregarAtributosTabu(aristasModificadas)}{
%     \SetAlgoLined
   \For{Arista a en aristasModificadas}{
	 	\hfill Ciclo: O(1)\\
	   atributosTabu.push(a) \hfill O(l$log(TenorTabu)$)\\
   }
  	
   \While{|atributosTabu| > TenorTabu}{
	   \hfill Ciclo: O(1)\\
	   atributosTabu.pop() \hfill O($log(TenorTabu)$)\\
   }
}


\end{algorithm}

Complejidad final: 
 O($CriterioParada$ * ( $n^5$ $\ast$ $log(TenorTabu)$ + $TenorTabu$)) =  O($CriterioParada$ * ( $(N+M)^5$ $\ast$ $log(TenorTabu)$ + $TenorTabu$)) \\
Siendo $N$ = cantidad pokeparadas y $M$ = cantidad gimnasios.\\

En general a la hora de implementar y experimentar, serán tomados porcentajes de $N+M$ para la cantidad de iteraciones límite y el tenor tabú ya que pueden ser valores muy grandes.\\

\subsubsection*{Detalles del algoritmo}
Justificación de la cota de complejidad del ciclo principal:
\begin{itemize}

\item $CriterioParada$: Este parametro determina cuantas veces ejecuta el ciclo principal de la busqurda tabú. Toma el mínimo valor entre $MaxIter$ y ($maxNoMejora$ - 1) * costoInicial($S_o$).
\item($maxNoMejora$ - 1) * costoInicial($S_o$) es cota superior de la cantidad de iteraciones ya que en el peor caso pueden pasar ($maxNoMejora - 1$) iteraciones sin que se mejore la solución. Una mejora, en peor caso es en una unidad. Como el costo es mayor a cero, se itera como mucho ($maxNoMejora$ - 1) * $costoInicial$ veces.
\item Luego, el algoritmo finaliza si se producen $maxNoMejora$ cantidad de iteraciones sin mejora de $costoInicial$ o si se alcanza $MaxIter$.  Como se ha comentado, pueden utilizarse independientemente y es así como serán evaluados en este informe.
\item En el peor caso $n = N+M$ es decir, que la solución inicial del algoritmo tiene todas las pokeparadas y gimnasios del mapa.
\item Arista = < Punto a , Punto b >
\item Punto = < entero x, entero y >
\item La función $calcularCosto$ ya fue definida en el apartado del algoritmo del ejercicio 3.
\item La función $vecindadFiltrada$ devuelve una lista de tuplas. Cada tupla posee una de las soluciones vecinas (obtenida a partir de algún movimiento: 2opt, 3opt o swap) y las aristas que se modificaron para llegar a esa solución. La soluciones han sido previamente "filtradas"; es decir, una solucion solo puede estar en esta lista si es una solución válida para nuestro problema. Generar vecindades 2opt y swap tiene complejidad O($n^3$), en cambio las vecindades 3opt toman O($n^4$).
	La cantidad de soluciones de cada vecindad está acotada de la misma manera que sus complejidades. En peor caso, se encontró una solucion válida en cada iteraci\'on.
\item La función $funcionAspiracion$ devuelve dada una vecindad y los atributos tabú, el recorrido que menos atributos tabú tenga y además las aristas modificadas para obtener ese recorrido. Implica iterar sobre la vecindad y sobre los atributos tabú. 
\item La funcion $tabuCount$ sirve para decidir si una solución es tabú o no y cuenta cuantos atributos tabú posee dando de esta manera una medida de "cuan tabú" es una solución. Se itera por las aristas de la solución en tiempo lineal.
\item $atributosTabu$ está implementado sobre el set proveido por la STL de c++. Se observa una estructura de arbol rojinegro que puede buscar, insertar y borrar en un tiempo O($log(TenorTabu)$), y se mantiene la cantidad de elementos de este por debajo de $TenorTabu$.
\end{itemize}
