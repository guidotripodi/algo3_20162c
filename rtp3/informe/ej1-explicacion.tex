Siendo el objetivo del maestro pokemon ganar todos los gimnasios recorriendo la menor distancia posible definimos como posible solución a una secuencia de lugares (gimnasios o pokeparadas) en donde sea posible pasar de un lugar al siguiente bajo las restricciones impuestas en el problema. La solución buscada propiamente dicha es aquella que forme el camino de menor longitud dentro de todas las soluciones posibles halladas.

Para esto se resuelve aplicar la técnica de backtracking sobre el conjunto total de caminos posibles. Las podas elegidas son las siguientes:

\begin{itemize}
\item Si la suma de pociones de todas las pokeparadas es inferior a la suma de pociones necesarias para derrotar a todos los gimnasios, entonces no hay solución.
\item No se pueden repetir lugares ya visitados: ésta es una restricción por consigna.
\item No se puede visitar un gimnasio sin la cantidad necesaria de pociones: de no contemplar esta poda se analizan caminos donde se pierde en gimnasios, los cuales por la primera restriccion, quedan como perdidos, y no generan solución.
\item No se recorren pokeparadas si la mochila está llena: el objetivo de una pokeparada es rellenar la mochila de pociones. Al estar llena la mochila, se desperdicia la pokeparada ya que no se la podrá volver a visitar. En adición, visitar una pokeparada innecesariamente, aumenta obligatoriamente la distancia recorrida por el maestro pokemon, ya que en vez de estar tomando un camino directo entre 2 lugares, accede a la pokeparada, aumentando así la distancia euclidea.
\item Los caminos se continúan construyendo siempre y cuando no superen la distancia mínima lograda anteriormente por alguna solución previa.
\item Si algún gimnasio requiere mayor cantidad de pociones que las que caben en la mochila, entonces no hay solución. 

\end{itemize}