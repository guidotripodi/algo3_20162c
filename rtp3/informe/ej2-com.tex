Como ya fué analizado, la longitud de cada solución (entendida como la cantidad de lugares visitados que la componen) se encuentra acotada por $n + m$ siendo $n$ la cantidad de gimnasios y $m$ las poke-paradas en cada instancia. El procedimiento, se aplica una vez por cada elemento, es decir O($n+m$) veces.

En cada una de estas veces, a diferencia del backtracking que analiza todo un sub-arbol de posibilidades, la heur\'istica golosa solo se queda con una sola rama. Con esto último podemos ver lo siguiente: agregar un elemento a una rama demanda O($n+m$), siendo la cantidad de elementos en ella O($n+m$). Su construcción se efectúa en O($(n+m)^2$) dejando la totalidad del procedimiento ($n+m$ ramas a construir) en una complejidad de O($(n+m)^3$).

Siendo que se siguen aplicando todas las podas que se efectuaban en el backtracking, el mejor caso de este algoritmo sucede al no haber solución, ya que la única operación que se realiza es el chequeo de la poda que preevalúa la instancia, con lo cual el algoritmo queda inferiormente acotado por $\Omega(n)$.