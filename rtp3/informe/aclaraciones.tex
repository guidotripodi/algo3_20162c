\subsection{Dificultades afrontadas}

A lo largo del desarrollo del trabajo nos encontramos con ciertos obstáculos que tuvimos que ir sorteando:
El primero de ellos fue el tiempo necesario para correr los tests: Al manejar un algoritmo de tiempo exponencial, como es el backtraking, tuvimos que limitarnos a un pequeño grupo de instancias para obtener las mediciones exactas para las comparaciones con otros algoritmos. La utilización de la optimización en la compilación de nivel 3 (-O3) ayudo en grán medida a correr los casos más grandes y poder agrandar un poco más la muestra.\\
A la hora de tomar las mediciones de los experimentos, nos vimos con el problema de la cantidad de instancias de un mismo tamaño a tomar: siendo que no podemos tomar la misma cantidad por cada tamaño debido a que acarrea un desbalance de exactitud (las instancias más chicas tendrían más datos muestrales que las grandes), tuvimos que adaptar la muestra al n(n+m) evaluado. Esto nos ocasionó un incremento substancial en la cantidad de instancias totales a evaluar, incrementando asi el tiempo de experimetación.\\

\subsection{Aclaraciones para correr las implementaciones}

Cada ejercicio fue implementado con su propio Makefile para un correcto funcionamiento a la hora de utilizar el mismo.\\
El ejecutable para el ejercicio 1 sera ej1 el cual recibir\'a como se solicito entrada por stdin y emitir\'a su respectiva salida por stdout.\\
Tanto el ejercicio 2 como el 3 y 4, compilar\'an de la misma forma y podr\'an ser ejecutados con ej2, ej3 y ej4 respectivamente.\\
A su vez, para poder chequear las mediciones de tiempo y nuestros casos de testeo se creo una carpeta nueva por cada ejercicio. Dentro de cada una se cuenta con el respectivo makefile que compila todos los test pudiendo as\'i probar cada uno por separado.