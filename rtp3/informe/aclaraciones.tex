\subsection{Dificultades afrontadas}

A lo largo del desarrollo del trabajo nos encontramos con ciertos obstáculos que tuvimos que ir sorteando:
El primero de ellos fue el tiempo necesario para correr los tests: Al manejar un algoritmo de tiempo exponencial, como es el backtraking, tuvimos que limitarnos a un pequeño grupo de instancias para obtener las mediciones exactas para las comparaciones con otros algoritmos. La utilización de la optimización en la compilación de nivel 3 (-O3) ayudo en grán medida a correr los casos más grandes y poder agrandar un poco más la muestra.\\
A la hora de tomar las mediciones de los experimentos, nos vimos con el problema de la cantidad de instancias de un mismo tamaño a tomar: siendo que no podemos tomar la misma cantidad por cada tamaño debido a que acarrea un desbalance de exactitud (las instancias más chicas tendrían más datos muestrales que las grandes), tuvimos que adaptar la muestra a un porcentaje del (n+m) evaluado. Esto nos ocasionó un incremento substancial en la cantidad de instancias totales a evaluar, incrementando asi el tiempo de experimetación.\\

\subsection{Aclaraciones para correr las implementaciones}

Cada ejercicio fué implementado con su propio Makefile para un correcto funcionamiento a la hora de utilizar el mismo.\\
El ejecutable para el ejercicio 1 sera ej1 el cual recibir\'a como se solicito entrada por stdin y emitir\'a su respectiva salida por stdout.\\
Tanto el ejercicio 2 como el 3 y 4, compilar\'an de la misma forma y podr\'an ser ejecutados con ej2, ej3 y ej4 respectivamente.\\
A su vez, se provee una carpeta extra por ejercicio con los tests realizados por el grupo en la experimentación. Los tests son totalmente funcionales, pero requieren de alguna modificación según la entrada creada. Dado que no hubo tiempo de generar un documento tutorial ni organizar mejor los fuentes, se recomienda al interesado en correr algún test, consultar a los desarrolladores para facilitar su uso.
Por otro lado, la mayoria de los casos testeados se encuentran en carpetas llamadas salidas-rtp o salidas-tp y las entradas generadas en carpetas como entradas-rtp y entradas-tp.