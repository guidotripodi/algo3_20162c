Nuestro algoritmo cuenta con 3 secciones marcadas. La primera, en la cual se chequean las podas sin soluci\'on, en la segunda se consigue el camino, y la \'ultima donde se devuelve dicho camino con la cantidad de nodos y distancia recorrida.\\
En detalle, para nuestra primer etapa, desarrollamos un ciclo para poder ver el poder que posee cada uno de los gimnasios, chequeando si la capacidad de la mochila soporta el poder de los mismos. A su vez, se va sumando el poder de cada gimnasio para luego chequear si la cantidad de pokeparadas alcanza para ganar en todos los gimnasios.\\
Luego, en la segunda secci\'on, se realiza el backtraking propiamente dicho, en donde un ciclo itera hasta obtener el camino con distancia recorrida \'optima. Dicho camino es armado tomando ciertas elecciones posibles y en caso que se encuentre alg\'un camino mejor se vuelve para atrás en las elecciones necesarias para tomar otro camino donde la distancia recorrida sea menor. Esto se realizar\'a hasta obtener el camino de distancia m\'inima.
Por \'ultimo, nuestra tercer secci\'on devolver\'a el camino obtenido con la distancia recorrida y la cantidad de nodos del camino.

Se muestra el pseudocodigo que ejemplifica lo comentado:

\begin{algorithm}[H]
\Fn{EJ1()}{
crear cola decisiones para guardar elecciones\\
eleccion contiene un id, posici\'on de la misma y distancia desde el punto anterior hasta el como tambien la cantidad de pociones hasta el momento\\
sePudoDeshacerEleccion $\gets$ verdadero \hfill O(1) \\
minimo $\gets$ -1 \hfill O(1)\\ 
\While{sePudoDeshacerEleccion}{  
	\If{lesGaneATodos(MaestroPokemon) $\wedge$ tiempo(MaestroPokemon) < minimo}{
	 \hfill  guarda: O(1)\\
			 minimo $\gets$ tiempo(MaestroPokemon) \hfill O(1)\\	
			 caminoRecorrido $\gets$ caminoRecorrido(MaestroPokemon) \hfill O(1)\\	
	}
			eleccion $\gets$ eleccionPosible(MaestroPokemon) \hfill O(n + m)\\			
			\If{$\exists$ eleccion}	{	
			aplicarEleccion(eleccion, MaestroPokemon)\hfill O(1)\\
		}\Else{
			sePudoDeshacerEleccion $\gets$ deshacerEleccion(MaestroPokemon) \hfill O(1)\\
		}
		}
		devolver minimo, tamaño(caminoRecorrido) y caminoRecorrido 	\hfill O(n + m)\\
		}
		
		\textbf{\hfill total: $O((n+m)!)$}\\
\end{algorithm}

\begin{algorithm}[H]
\Fn{aplicarEleccion(eleccion, maestroPokemon)}{
	encolar en decisiones la eleccion tomada  \hfill O(1) \\
	actualizar estado del sistema en base a la decision tomada: \hfill O(1) \\
	\If{tipo(eleccion) == GIMNASIO}{
	\hfill  guarda: O(1)\\
	 	decrementar	maestroPokemon.cantidadGymFaltantes en 1 \hfill O(1) \\
		 maestroPokemon.cantidadPociones  $\gets$ pociones(maestroPokemon) - pocionesNecesarias(eleccion) \hfill O(1) \\
	}\Else{
		 maestroPokemon.cantidadPociones $\gets$ pociones(maestroPokemon) + 3\hfill O(1) \\
	
	}
	sumar a maestroPokemon.tiempo tiempo(eleccion) \hfill O(1) \\
	guardar estado como efectuado\hfill O(1) \\
	actualizar eleccionActual con la nueva eleccion\hfill O(1) \\

}
\textbf{\hfill Complejidad total: $O(1)$}\\ 
\end{algorithm}


\begin{algorithm}[H]
\Fn{deshacerUltimaEleccion(maestroPokemon)}{
	\If{hay decisiones para desencolar}{
	desencolar de decisiones una eleccion\hfill O(1) \\
	actualizar estado del sistema en base a la decision desencolada: \hfill O(1) \\
	\If{tipo(eleccion) == GIMNASIO}{
	\hfill  guarda: O(1)\\
	 	incrementar	maestroPokemon.cantidadGymFaltantes en 1 \hfill O(1) \\
		 maestroPokemon.cantidadPociones $\gets$ pociones(maestroPokemon) + pocionesNecesarias(eleccion) \hfill O(1) \\
	}\Else{
		 maestroPokemon.cantidadPociones $\gets$ pociones(maestroPokemon) - 3\hfill O(1) \\
	
	}
	restar a maestroPokemon.tiempo tiempo(eleccion) \hfill O(1) \\
	actualizar eleccionActual con la eleccion anterior a la que acabamos de sacar \hfill O(1) \\

	devolver verdadero \hfill O(1) \\	
	}\Else{ 
		devolver falso \hfill O(1) \\	
	}
}
\textbf{\hfill Complejidad total: $O(1)$}\\ 
\end{algorithm}


\begin{algorithm}[H]
\Fn{eleccionPosible(maestroPokemon)}{
	\While{Hay elecciones disponibles}{
	\hfill Ciclo: O(n+m)\\
		\If{eleccion.tipo == POKEPARADA}{
		\hfill  guarda: O(1)\\
			\If{maestroPokemon.pociones == maestroPokemon.capacidadMochila}{
			\hfill  guarda: O(1)\\
				 recalcularEleccion en base a ultima posicion correcta \hfill O(1) \\	
				}\Else{
					devolver eleccion \hfill O(1) \\
				}			
			}\Else{
			\If{maestroPokemon.pociones < pocionesNecesarias(eleccion)}{
				\hfill  guarda: O(1)\\
				recalcularEleccion en base a ultima posicion correcta \hfill O(1) \\	
				}\Else{
					devolver eleccion \hfill O(1) \\
				}
			
			
			}	
	
	

	}
}
\textbf{\hfill Complejidad total: $O(n+m)$}\\ 
\end{algorithm}

\subsubsection*{Estructura interna para chequear elecciones y estados ocurridos}

Se trabajó con dos estructuras internas denominadas $MaestroPokemon$ y $Eleccion$, donde la primera tiene la informaci\'on sobre la cantidad de gimnasios y pokeparadas, la cantidad de gimnasios ya derrotados, la posici\'on actual, la cantidad de posiciones que posee en cada momento como tambi\'en la capacidad m\'axima de la mochila, y dos arreglos los cuales poseen todos los destinos posibles visitados o no y otra con solo los visitados.
Mientras que nuestra segunda estructura, como el nombre lo indica, nos otorga toda la informaci\'on necesaria para poder realizar la elecci\'on, como es el id de la misma para no ser repetida, la posici\'on en donde se encuentra, la cantidad de pociones necesarias en caso de ser un gimnasio y la distancia para llegar a dicha posici\'on desde cierto punto.

