\begin{algorithm}[H]
\caption{}
\begin{algorithmic}[1]
\Function{EJ2 - Algoritmo Kruskal}{\param{in}{}{}}{$\ensuremath{\rightarrow}$ \param{out}{} {}}
\state creo UF uf(n) \hfill //O(1)
\state creo entero u,v \hfill //O(1)
\While{i < e} \hfill //O($E$)
\state u $\gets$ grafo[i].Second.First \hfill //O(1)
\state u $\gets$ grafo[i].Second.Second \hfill //O(1)
\If{$\neg$ uf.(find(u,v))} \hfill //O(N)
\state uf.unir(u,v) \hfill //O(N)
\state T $\gets$ T + grafo[i].First \hfill //O(1)
\endif
\state i++ \hfill //O(1)
\endwhile
\EndFunction 
\end{algorithmic}
\hrule
\complejidad{  }
\end{algorithm}

\begin{algorithm}[H]
\caption{}
\begin{algorithmic}[1]
\Function{EJ2 - Find}{\param{in}{P}{entero}}{$\ensuremath{\rightarrow}$ \param{out}{S} {entero}}
\state creo entero root con P \hfill //O(1)
\While{root $\neq$ id[root]} \hfill //O(1)
\state root $\gets$ id[root] \hfill //O(1)
\endwhile
\While{p $\neq$ root} \hfill //O(1)
\state creo entero nuevoP con id[P] \hfill //O(1)
\state id[P] $\gets$ root \hfill //O(1)
\state P $\gets$ nuevoP \hfill //O(1)
\endwhile
\state devolver root \hfill //O(1)
\EndFunction 
\end{algorithmic}
\hrule
\complejidad{??}
\end{algorithm}

\begin{algorithm}[H]
\caption{}
\begin{algorithmic}[1]
\Function{EJ2 - Unir}{\param{in}{X}{entero}\param{in}{Y}{entero}}{$\ensuremath{\rightarrow}$ \param{out}{S} {entero}}
\state creo entero i con find(X) \hfill //O(1)
\state creo entero j con find(Y) \hfill //O(1)
\If{tamaño(i) < tamaño(j)} \hfill //O(1)
\state id[i] $\gets$ j\hfill //O(1)
\state size[j] $\gets$ size[j] + size[i] \hfill //O(1)
\Else
\state id[j] $\gets$ i\hfill //O(1)
\state size[i] $\gets$ size[i] + size[j] \hfill //O(1)	
\endif
\state contador-- \hfill //O(1)	
\EndFunction 
\end{algorithmic}
\hrule
\complejidad{UV}
\end{algorithm}


\textbf{Aclaraciones de variables y transformaci\'on de entrada}

\begin{itemize}
\item {\bf Tipo UF}: Dicho tipo, fue creado para contener m\'as informaci\'on del uni\'on find, como el id, la cantidad y el tamaño, como as\'i tambi\'en las operaciones unir buscar y ver si estan o no conectados.
\item {\bf Tipo nodo}: Dicho tipo, es una tupla de enteros.
\item {\bf Tipo arista}: Dicho tipo, es una tupla de entero y nodo.
\item {\bf Variable grafo}: Dicha variable es un vector de arista
\end{itemize}  