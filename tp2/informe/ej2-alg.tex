\begin{algorithm}[H] %or another one check
 \caption{Algoritmo EJ2}
 %     \SetAlgoLined
	representanteFinal = -1 //QUEDA MAL PONER -1

	\For{Arista ar $\exists$ aristas}{
		\If{Buscar(ar.inicio) == Buscar(ar.fin)}{
			Unir(ar.inicio, ar.fin, ar.costo);
		}

		\If{representanteFinal $\geq$ 0}{
			finalizar ciclo
		}
	}

	\If{representanteFinal $\geq$ 0}{
		 devolver costCompLider[representanteFinal]
	}\Else{
         devolver sinSolucion
    }
\end{algorithm}


\begin{algorithm}[H] %or another one check
 \caption{Funcion Buscar(entero x)}
 %     \SetAlgoLined
 	\If{padre[x] == x}{devolver x}{
 	para entero p $\gets$ Buscar(padre[x])\\
	para padre[x] asignar p\\
	devolver p\\
 	}
\end{algorithm}

/ESTO ESTA MAS IMPLEMENTATIVO QUE ANTES HAY HASTA ; JAJA DEJO LO Q ESTABA ANTES


\begin{algorithm}[H] %or another one check
 \caption{Funcion Unir(enteros x, y, costo)}
 %     \SetAlgoLined
 	x = Buscar(x); y = Buscar(y);\\
    
	\If{altura[x] == altura[y]}{
		altura[x]++;\\
		padre[x] = y;\\
        cantAristas[y] = cantAristas[y]+cantAristas[x]+1;\\
        costCompLider[y] + = costCompLider[x]+costo;\\
        
        \If{cantAristas[y] == $\#$nodos-1} {
            representanteFinal = y;\\
        }
	}\eIf{altura[x] < altura[y]}{
		padre[x] = y;\\
        cantAristas[y] = cantAristas[y]+cantAristas[x]+1;\\
        costCompLider[y] + = costCompLider[x]+costo;\\
        \If {cantAristas[y] == $\#$nodos-1} {
            representanteFinal = y;\\
        }
	}{
		padre[y] = x;\\
        cantAristas[x] = cantAristas[x]+cantAristas[y]+1;\\
        costCompLider[x] + = costCompLider[y]+costo;\\
        
        \If{cantAristas[x] == $\#$nodos-1} {
            representanteFinal = x;\\
        }
	}

\end{algorithm}

//EL QUE ESTABA


\begin{algorithm}[H] %or another one check
 \caption{Funcion Unir(enteros X, Y, COSTO)}
 %     \SetAlgoLined
 	para entero x asignar Buscar(X)\\
 	para entero y asignar Buscar(Y)\\
	\eIf{altura[x] $\leq$ altura[y]} {
	 	para padre[x] asignar y\\
 	para cantidadAristas[y] asignar cantidadAristas[y] + cantidadAristas[x] +1\\
 	para costo[y] asignar costo[y] + costo[x] + COSTO\\
 	\If{altura[x] == altura[y]}{
 	 	para altura[x] incrementar 1 }
	}{ 	 	para padre[y] asignar x\\
 	para cantidadAristas[x] asignar cantidadAristas[y] + cantidadAristas[x] +1\\
 	para costo[x] asignar costo[x] + costo[y] + COSTO\\	}
 
\end{algorithm}

\textbf{Aclaraciones de variables y transformaci\'on de entrada}

\begin{itemize}

\item {\bf Arreglo cantAristas}: guarda la cantidad de aristas que tiene el representante de una componente conexa
\item {\bf Arreglo costCompLider}: guarda el costo total de una componente conexa por representante 
\item {\bf Arreglo padre}: para cada nodo, quien es su representante, es decir, en que componente conexa se encuentra
\item {\bf Arreglo altura}: indica la cantidad de nodos de una componente conexa por representante
\end{itemize}  