A continuaci\'on se detalla el pseudo-c\'odigo de la parte principal del algoritmo:

\begin{algorithm}[H] %or another one check
 \caption{Algoritmo EJ3}
 %     \SetAlgoLined
 	para Grafo fortaleza asignar grafo(entrada, entrada1,entrada2)\\
 	\For{nodo $\in$ fortaleza}{
 	para nodo.distancia asignar invalido\\
 	para nodo.visitado asignar falso\\
 	}
 	\While{actual $\neq$ masCercano(visitados, distancia, cantidadNodos)}
 	{
	para visitados[actual] asignar verdadero\\
	para vecinos asignar vecinos(fortaleza,actual)\\
	\For{arista $\in$ vecinos}{
	para dist asignar distancia(actual) + peso(arista)\\
	\If{ dist < distancia(arista.destino) || distancia(arista.destino) == invalido}{
	para distancia[v.destino] asignar dist\\
	para previos[v.destino] asignar actual\\
	}
	}
 	}
\end{algorithm}


\textbf{Aclaraciones de variables y funciones}
\begin{itemize}
\item La funci\'on grafo, crea grafo uniendo las aristas con los respectivos nodos que son adyacentes.
\item La funci\'on vecinos devuelve una lista de aristas correspondiente a los nodos adyacentens al v\'ertice consultado
\end{itemize}
