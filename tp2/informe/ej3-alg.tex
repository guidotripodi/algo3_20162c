A continuaci\'on se detalla el pseudo-c\'odigo de la parte principal del algoritmo:

\begin{algorithm}[H] %or another one check
 \Fn{Dijsktra(estaciones, vias)}{
 %     \SetAlgoLined
 	para Grafo fortaleza asignar grafo(estaciones,vias)\hfill O(N)\\
 	\For{nodo $\in$ fortaleza}{
 	\hfill  Ciclo: O(N??)\\
 	para nodo.distancia asignar invalido\hfill O(1)\\
 	para nodo.visitado asignar falso\hfill O(1)\\
 	}
 	\While{$\exists$ nodo alcanzable que no haya sido visitado}
 	{\hfill  Ciclo: O(N??)\\
	para actual asignar masCercano(visitados, distancia, cantidadNodos)
	para visitados[actual] asignar verdadero\hfill O(1)\\
	para vecinos asignar vecinos(fortaleza,actual)\hfill  O(??)\\
	\For{arista $\in$ vecinos}{
	\hfill  Ciclo: O(N??)\\
	para dist asignar distancia(actual) + peso(arista)\hfill O(1)\\
	\If{ dist < distancia(arista.destino) || distancia(arista.destino) == invalido}{
	para distancia[v.destino] asignar dist\hfill O(1)\\
	para previos[v.destino] asignar actual\hfill O(1)\\
	}
	}
 	}
 	}
\end{algorithm}

\textbf{Complejidad total del algoritmo: O($N^2$ $\times$ $log^2(N)$} con N = cantidad de estaciones.

\textbf{Aclaraciones de variables y funciones}
\begin{itemize}
\item La funci\'on grafo, crea grafo uniendo las aristas con los respectivos nodos que son adyacentes.
\item La funci\'on vecinos devuelve una lista de aristas correspondiente a los nodos adyacentens al v\'ertice consultado
\item La funci\'on masCercano devuelve el nodo con la menor distancia al origen, que no haya sido visitado. 
\end{itemize}
