\indent Para verificar el correcto funcionamiento de nuestro algoritmo , elaboramos disversos tests,
los cuales ser\'an enunciados a continuaci\'on.\\

\begin{center}
 \textbf{Todos los arqueologos y canibales presentan la misma velocidad}
\end{center}

Este caso se da cuando V$_{i}$ = W$_{j}$ $\forall$ (i,j)  $\gets$ [0..6] \\

 Para este tipo de testeo mostraremos a continuaci\'on un ejemplo del mismo, exponiendo su respectivo resultado. Adem\'as veremos m\'as adelante, que por el desarrollo de nuestro algoritmo y sus respectivas podas este ser\'a el peor caso en referencia a la perfomance del mismo.\\
 
 Con un:
 
$Cantidad$ $de$ $arqueologos:$ $4 $

$Cantidad$ $de$ $canibales:$ $2 $ 

$Velocidad$ $de$ $arqueologos:$ $10$ $10$ $10$ $10$

$Velocidad$ $de$ $canibales:$ $10$ $10 $

  Obtuvimos el siguiente resultado:

$Velocidad$ $de$ $cruce$ $total: $ $90$\\



 \begin{center}
 \textbf{No hay canibales}
\end{center}

Esta versi\'on se da cuando M = 0. 

Para este tipo de testeo mostraremos a continuaci\'on un ejemplo del mismo, exponiendo su respectivo resultado.\\

 Con un:

$Cantidad$ $de$ $arqueologos:$ $5 $  

$Cantidad$ $de$ $canibales:$ $0 $ 

$Velocidad$ $de$ $arqueologos:$ $15$ $10$ $5$ $2$ $20$ 

$Velocidad$ $de$ $canibales: $  

  Obtuvimos el siguiente resultado:

$Velocidad$ $de$ $cruce$ $total: $ $56$


\begin{center}
 \textbf{Todos los arqueologos y canibales presentan velocidades distintas}
\end{center}

Este caso se da cuando V$_{i}$ $\neq$ W$_{j}$ $\forall$ (i,j)  $\gets$ [0..6] 

Aqu\'i veremos, un ejemplo del conjunto de test de este tipo, exponiendo su respectivo resultado.\\

 Con un:

$Cantidad$ $de$ $arqueologos: 3 $ 

$Cantidad$ $de$ $canibales: 2 $ 

$Velocidad$ $de$ $arqueologos:$ $2$ $4$ $6 $

$Velocidad$ $de$ $canibales:$ $1$ $3$ $5 $

  Obtuvimos el siguiente resultado:

$Velocidad$ $de$ $cruce$ $total: $ $18$\\


\begin{center}
 \textbf{Hay un canibal cada dos arqueologos}
\end{center}

Este tipo de caso se cumple cuando N = 2 $\ast$ M

Un ejemplo que simboliza el conjunto de test de este tipo es el siguiente:\\

 Con un:

$Cantidad$ $de$ $arqueologos:$ $4 $

$Cantidad$ $de$ $canibales:$ $2$

$Velocidad$ $de$ $arqueologos:$ $3$ $6$ $9$ $12$

$Velocidad$ $de$ $canibales:$ $1$ $2$ 


  Obtuvimos el siguiente resultado:

$Velocidad$ $de$ $cruce$ $total: $ $33$



\begin{center}
 \textbf{Hay mas canibales que arqueologos}
\end{center}

Este tipo de caso se cumple cuando N < M

A continuaci\'on enunciaremos, un ejemplo del conjunto de test de este tipo, exponiendo su respectivo resultado.\\

 Con un:

$Cantidad$ $de$ $arqueologos:$ $2 $

$Cantidad$ $de$ $canibales:$ $3$

$Velocidad$ $de$ $arqueologos:$ $3$ $6$

$Velocidad$ $de$ $canibales:$ $1$ $2$ $5$

  Obtuvimos el siguiente resultado:

$Velocidad$ $de$ $cruce$ $total:$ \textit{NO HAY SOLUCI\'ON} \\


