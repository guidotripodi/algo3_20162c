Siendo la entrada una matriz de $F \times C = N$, en donde cada celda es un nodo en potencia (es decir si no hubiese paredes) entonces el grafo de mayor tamaño tiene N nodos. Cada baldosa puede recorrerse en 2 direciones y 4 sentidos distintos, con lo cual cada nodo es de grado 4. El algoritmo BFS original, por su parte, tiene una complejidad de O($|V|+|E|$), siendo V el conjunto de nodos y E el de aristas del grafo que se le pasa como parámetro, con lo cual, de aplicar BFS sobre el grafo de nuestro problema, la coplejidad sería O($F \times C$).

No obstante, al modificar el BFS permitiendo que se puedan visitar nodos ya visitados, la complejidad se ve afectada. El algoritmo propuesto permite volver a recorrer un nodo, siempre y cuando la cantidad de paredes derribadas para llegar a él, sea mejor a la anterior cantidad lograda; es decir que cada nodo que se vuelve a recorrer, en potencia, se recorre P veces (siendo P la cantidad de paredes a romper del enuncaido). Habiendo N nodos, se resuelve el algoritmo en O($N \times P$) = O($F \times C \times P$).