\indent Para corroborar el correcto funcionamiento de nuestro algoritmo implementado desarrollamos los siguientes tests:\\


\begin{center}
 \textbf{Sin Soluci\'on}
\end{center}

Este caso se da cuando el nodo que representa a la estaci\'on final queda aislado  \\


\begin{center}
 \textbf{Sin ejes}
\end{center}

Esta familia de casos se da cuando todos los nodos que representa a la estaciones quedan aislados  \\

\begin{center}
 \textbf{Camino simple}
\end{center}

Este caso se da cuando el grafo que se recibe como par\'ametro ya presenta un camino simple armado desde la estaci\'on inicial a la final

\begin{center}
 \textbf{Multiples caminos de igual peso llegan a destino}
\end{center}

Este caso se da cuando existen varios ramas dentro del grafo que van desde el nodo inicial hasta el \'ultimo y la suma de dichos pesos termina siendo igual. Veremos m\'as adelante que por la implementaci\'on y desarrollo de nuestro algoritmo este terminar\'a siendo el peor caso.

\begin{center}
 \textbf{Caso random}
\end{center}

Este caso se da cuando se recibe grafos con aristas y nodos sin ninguna particularidad.

