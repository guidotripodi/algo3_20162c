La entrada de nuestro algoritmo tiene $m$ lineas. Estas representan las aristas que va a contener nuestro grafo. Cada arista es procesada y almacenada en nuestro grafo en tiempo constante. Sabemos que un digrafo tiene la cantidad de aristas acotadas por $n*(n-1)$ siendo $n$ la cantidad de nodos. Entonces construir nuestro grafo tiene una complejidad de $O(n^2)$.

Para el algoritmo de Dijkstra implementamos su cola de prioridad con dos arreglos, uno con la minima distancia encontrada hacia el nodo y otro que nos indica si un nodo fue visitado o no. Por lo tanto, conseguir el proximo nodo a visitar es recorrer uno por uno los elementos del arreglo distancia y quedarnos con el indice del nodo con menor valor valido siempre que este marcado como "no visitado" en el otro arreglo. Esta operacion tiene costo lineal en  la cantidad de nodos ($O(n)$) y se realiza en el peor caso n veces caundo el ultimo nodo que visite es la salida, lo cual resulta en un costo de $O(n^2)$. 

Luego se recorren los vecinos de un nodo y se actualizan sus distancias. Dada nuestra representacion en arreglos, actualizar la distancia de cada vecino toma tiempo constante. Para los $n$ nodos se recorren sus vecinos, que en principio podrian ser $n-1$. Entonces, la complejidad de esta operación es $O(n^2)$.

Finalmente recorremos en complejidad $O(n)$ un arreglo ($prev$) donde veníamos guardando el vecino desde donde se obtuvo la distancia minima para el nodo. De esta manera construimos efectivamente el camino minimo desde el destino hacia el origen y lo imprimimos. 

Sumando estas operaciones nuestra complejidad final es:

$O(n^2) + O(n^2) + O(n^2) + O(n) = O(n^2)$