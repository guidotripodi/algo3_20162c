En esta ocasión se debe encontrar, dada una serie de habitaciones, la forma menos costosa de unirlas, teniendo que derribar determinadas paredes para lograrlo. Dada 2 habitaciones: el muro menos costoso a derribar que une a las mismas será la elección a tomar para formar la solución, es decir, se deben elegir todos los muros menos costosos que unen las distintas habitaciones para obtener la soluci\'on buscada.

Un planteo de modelo sobre grafo caracteriza a cada habitación como un conjunto de puntos caminables y cada uno de estos como un nodo adyacente a los puntos de la misma habitaci\'on el cual posee, siendo H la cantidad total de habitaciones en el mapa, H componentes conexas. Lo que se busca entonces es una forma de unir estas componentes de la forma menos costosa. Para esto unimos a 2 habitaciones separadas por una pared con un eje de peso igual al esfuerzo de derribarla. A las aristas dentro de cada componente les asignamos el peso nulo.

Para obtener la soluci\'on, se busca el AGM del grafo planteado mediante el algoritmo de Kruskal. A dicho grafo resultante del algoritmo se le calcula el peso, el cual representa la respuesta al problema.

Para desarrollar el algoritmo de b\'usqueda del AGM del grafo, implementamos las funciones find y union, las cuales se encargan de buscar al padre del nodo consultado en el arbol y unir al nodo y/o arbol que presente menor altura al mayor, de esta forma se ir\'a confeccionando el arbol final, modificando a los padres de los nodos y la respectiva altura del mismo como asi tambi\'en la cantidad de aristas y costos de estas que ir\'a teniendo dicho arbol a medida que se le ir\'an uniendo nuevos arboles.

