En esta ocasión se debe encontrar, dada una serie de habitaciones, la forma menos costosa de unirlas, teniendo que derribar determinadas paredes para lograrlo. Dada 2 habitaciones: el muro menos costoso para derribar que las une será la elección a tomar para formar la solución, es decir, se deben elegir todos los muros menos costosos que unen las distintas habitaciones para obtener la solucion buscada.

Un planteo de modelo sobre grafo caracteriza a cada habitación como un conjunto de puntos caminables y cada uno de estos como un nodo adyacente a los puntos de la misma habitacion el cual posee, siendo H la cantidad total de habitaciones en el mapa, H componentes conexas. Lo que se busca entonces es una forma de unir estas componentes de la forma menos costosa. Para esto unimos a 2 habitaciones separadas por una pared con un eje de peso igual al esfuerzo de derribarla. A las aristas dentro de cada componente les asignamos el peso nulo.

CLAVAR EL DRAWING

Para obtener la solucion, se busca el AGM del grafo planteado mediante el algoritmo de Kruskal; a él se le calcula el peso, el cual representa la respuesta al problema.

