\begin{algorithm}[H]
\caption{Atravezando el laberinto}
\begin{algorithmic}[1]
\Function{EJ1 - Algoritmo BFS}{\param{in}{ }{Integer}, \param{in}{ }{List<Integer>}}{$\ensuremath{\rightarrow}$ \param{out}{res}{Integer}}
\state cola<Nodo> encolo nodoInicial \hfill //O(log N)
\while{cola $\neq$ VACIO} \hfill //O($N \ast M$)
\state creo Nodo* actual $\gets$ tope(cola) \hfill //O(1)
\state desencolo cola \hfill //O(1)
\state actual.visitado $\gets$ Verdadero \hfill //O(1)
\state creo entero i con actual.i \hfill //O(1)
\state creo entero j con actual.i \hfill //O(1)
\state procesoNodo(i-1, j, actual) \hfill //O(1??)
\state procesoNodo(i+1, j, actual) \hfill //O(1??)
\state procesoNodo(i, j-1, actual) \hfill //O(1??)
\state procesoNodo(i,j+1, actual) \hfill //O(1??)
\endwhile
\EndFunction 
\end{algorithmic}
\hrule
\complejidad{$N \ast M \ast P $}
\end{algorithm}


\begin{algorithm}[H]
\caption{Atravezando el laberinto}
\begin{algorithmic}[1]
\Function{EJ1 - ProcesoNodo}{\param{in}{i}{Integer}, \param{in}{j}{Integer}\param{in}{actual}{Nodo$\*$}}{$\ensuremath{\rightarrow}$ \param{out}{}{}}
\If{i > 0 $\wedge$ i < filas $\wedge$ j > 0 $\wedge$ j < columnas} \hfill //O(1)
\state creo Nodo* node $\gets$ Map[i][j] \hfill //O(1)
\If{$\neg$visitado(node)} \hfill //O(1)
\If{$\neg$EsPared(node) $\wedge$ $\neg$EsPared(actual)} \hfill //O(1)
\If{cantParedes(node) ==-1 $\vee$ \state
cantParedes(actual) + EsPared(node) <  cantParedes(node)} \hfill //O(1)
\state node.distancia $\gets$ distMinANodo(actual) + 1  \hfill //O(1)
\state node.cantidadParedesRotas $\gets$ cantParedes(actual) + EsPared(node)\hfill //O(1)
\If{$\neg$marcado(node)} \hfill //O(1)
\state node.marcado $\gets$ VERDADERO  \hfill //O(1)
\state cola encolar (node)  \hfill //O(LOG N)
\endif
\endif
\endif
\endif
\endif
\EndFunction 
\end{algorithmic}
\hrule
\complejidad{$????$}
\end{algorithm}

\textbf{Aclaraciones de variables y transformaci\'on de entrada}

\begin{itemize}
\item {\bf Tipo Nodo}: Dicho tipo, fue creado para contener m\'as informaci\'on de cada nodo, el mismo esta compuesto por posici\'on i, j del nodo en la matriz, si es pared o no, y si esta marcado o visitado y por \'ultimo la cantidad de paredes destruidas y la distancia recorrida hasta dicho nodo.
\item {\bf Variable Map}: Dicha variable, es creada global como una matriz $(N \ast M)$ de nodos.
\item {\bf Variable cola}: Como la palabra lo indica, en esta cola se ir\'an encolando los nodos.
\end{itemize}  


