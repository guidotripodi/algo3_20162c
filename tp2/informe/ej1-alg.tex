\begin{algorithm}[H] %or another one check
 \Fn{main()}{
 %     \SetAlgoLined
	encolo nodo origen en una cola de nodos \hfill O(log (F $\ast$ C))\\
	    \While{$\neg$ cola vacia}{ \hfill Ciclo: O(F $\ast$ C $\ast$ P)\\
	    Nodo actual $\gets$ tope cola \hfill O(1)\\
	    procesar vecinos\hfill O(??)\\
	    }
}
\end{algorithm}


\begin{algorithm}[H] %or another one check

\Fn{procesarNodo(Nodo nod)}{
 	Nodo nod es un vecino de actual en el Mapa\\
	
       	   	 \If{$\#$ paredes rotas hasta actual + esPared(nod) < $\#$ paredes rotas hasta actual nod}{
       	   	 para distancia hasta nod $\gets$ distancia hasta actual + 1 \hfill O(1)\\
       	   	 para $\#$ paredes rotas hasta nod $\gets$ $\#$ paredes rotas hasta actual + esPared(nod)\hfill O(1)\\ 
			\If{nod = nodo destino}	{
				devolver distancia hasta nodo destino y terminar \hfill O(1)
			}
	   	   	 \If{$\#$ paredes rotas hasta nod $\leq$ Pmax}{
       	   		 encolar nod \hfill O(log (F $\ast$ C)??)
       	   		 }
       	   } 
			
}
\end{algorithm}

\textbf{Complejidad total del algoritmo: O(F $\times$ C $\times$ P)}

\textbf{Aclaraciones de variables y transformaci\'on de entrada}

\begin{itemize}
\item {\bf Variable Mapa}: la matriz $(F \ast C)$ de nodos.
\item {\bf Tipo Nodo}: Dicho tipo, fue creado para contener m\'as informaci\'on de cada nodo, el mismo esta compuesto por posici\'on i, j del nodo en el Mapa, si es pared o no y por \'ultimo la cantidad de paredes destruidas y la distancia recorrida hasta dicho nodo.
\item {\bf cola}: Como la palabra lo indica, en esta cola se ir\'an encolando los nodos.
\end{itemize}  



