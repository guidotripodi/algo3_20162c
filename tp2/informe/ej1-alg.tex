\begin{algorithm}[H] %or another one check
 \caption{Algoritmo EJ1}
 %     \SetAlgoLined
	encolo nodoInicial en una cola de nodos \\
	    \While{$\neg$ vacia(cola)}{
	    nodo actual $\gets$ tope(cola) \\
	    desencolo(cola) \\
	    para visitado(actual) asignar Verdadero \\
	    procesarNodo(actual, posicion del nodo)\\
	    }
	    
\end{algorithm}


\begin{algorithm}[H] %or another one check
 \caption{procesarNodo}
 	nodo node $\gets$ Map[posicionDelNodoActual]\\
	\If{$\neg$ visitado(node)}{
       	   	 \If{$\neg$ esPared(node) $\wedge$ $\neg$EsPared(actual)}{
       	   	 \If{cantidadParedes(actual) + esPared(node) < cantidadParedes(node)}{
       	   	 para distancia(node) asignar distanciaMinima(actual) + 1\\
       	   	 para cantidadParedes(node) asignar cantidadParedes(actual) + esPared(node) \\
       	   	 \If{$\neg$ marcado(node)}{
       	   	 para marcado(node) asignar verdadero\\
       	   	 meter en cola a node\\
       	   	 }
       	   	 }
             
			}
			}
\end{algorithm}


\textbf{Aclaraciones de variables y transformaci\'on de entrada}

\begin{itemize}
\item {\bf Tipo Nodo}: Dicho tipo, fue creado para contener m\'as informaci\'on de cada nodo, el mismo esta compuesto por posici\'on i, j del nodo en la matriz, si es pared o no, y si esta marcado o visitado y por \'ultimo la cantidad de paredes destruidas y la distancia recorrida hasta dicho nodo.
\item {\bf Variable Map}: Dicha variable, es creada global como una matriz $(N \ast M)$ de nodos.
\item {\bf Variable cola}: Como la palabra lo indica, en esta cola se ir\'an encolando los nodos.
\end{itemize}  


