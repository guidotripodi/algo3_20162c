El algoritmo BFS permite saber el menor recorrido desde un nodo hasta todos los demas. Al alcanzar un nodo no visitado, el algoritmo asegura que es el camino más rápido al mismo. El algoritmo propuesto, siguiendo este invariante, visita por primera vez a cada nodo por el camino más directo, y al llegar a destino se interrumpe para devolver el tiempo mínimo.

Ya que no necesariamente el camino más corto es uno factible (como los vistos en la representación de la explicación) se permite que los otros caminos, que requieran utilizar los nodos ya visitados por aquellos más directos, puedan reutilizarlos si es que para llegar a cada nodo, se rompieron menos paredes: De haber roto más no tiene sentido seguir analizando la ruta, ya que llegará a menos nodos que aquel que recorrió al nodo previamente, y de haber roto igual cantidad de paredes quiere decir que el anterior camino llegó en menor tiempo al nodo analizado). De esta manera se pueden explorar todos los caminos posibles hasta encontrar el que nos lleve a la solución. Con estas condiciones, de llegar al nodo destino, será por el camino factible más corto.