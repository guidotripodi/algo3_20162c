Para demostrar que nuestro algoritmo es v\'alido para solucionar el problema dado tendremos que probar las siguientes condiciones:\\

\begin{itemize}
\item Si el grafo analizado tiene solución desde un nodo origen a un nodo destino dentro del limite de paredes a romper, el algoritmo siempre termina y si no posee solución solución tambien finaliza dado que no existen ciclos infinitos.
\item El camino resultante es el m\'as corto dentro de los posibles.
\end{itemize}

Como nuestro algoritmo puede encolar nodos ya visitados queremos ver que no se producen ciclos infinitos entre nodos visitados.

Dado que el algoritmo no encola nodos de los bordes del grafo, no existirá la posibilidad de generarse ningún ciclo llegando a un borde.

Supongamos que se producen ciclos infinitos entre nodos ya visitados. 

Sea N$\_$1 el nodo visitado del que se parte, con una cierta cantidad de paredes rotas y C el ciclo de nodos visitados desde N$\_$1 hasta N$\_$k que se vuelven a actualizar con menor cantidad de paredes derribadas (dado que si esto no sucede, no hay circuito porque no se actualizan todos los nodos del ciclo mencionado). 
La cantidad de paredes del camino desde N$\_$1 hasta N$\_$k es creciente en cantidad de paredes rotas. Al intentar ir de N$\_$k a N$\_$1, lo que sucede es que la cantidad de paredes rotas N$\_$k + 1 o + 0 (si N$\_$1 es pared o no) deberia ser menor a la cantidad de paredes rotas en N$\_$1. Pero esto no es cierto dado que sabemos que cualquier camino es creciente en cantidad de paredes o se mantiene con la misma cantidad de paredes rotas. Por lo tanto la cantidad de paredes rotas en N$\_$k es mayor que N$\_$1 y el ciclo no podrá cerrarse. Absurdo!! que viene de suponer que se producen ciclos infinitos entre nodos visitados.

Por lo tanto, no habrá ciclos infinitos entre nodos visitados.

Si hay solucion, no se excede la cantidad de paredes rotas para llegar al destino, por lo tanto el algoritmo encolará todos los nodos y en particular aquellos que estén conectados al mismo. Entonces,el destino será actualizado y el algoritmo finalizará su ejecucion



