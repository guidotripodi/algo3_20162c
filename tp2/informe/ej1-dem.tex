Para demostrar que nuestro algoritmo es v\'alido para solucionar el problema dado tendremos que probar las siguientes condiciones:\\

\begin{itemize}
\item Si el grafo analizado tiene solución desde un nodo origen a un nodo destino dentro del limite de paredes a romper, el algoritmo siempre termina y si no posee solución solución tambien finaliza dado que no existen ciclos infinitos.
\item El camino resultante es el m\'as corto dentro de los posibles
\end{itemize}


PARA EL PUNTO 1: Como nuestro algoritmo puede encolar nodos ya visitados queremos ver que no se van a producir ciclos infinitos por lo enunciado.
Si existe un camino que llega a un nodo ya visitado, dicho camino es aquel que requiere una mayor cantidad de pasos para llegar a \'el. A partir de aqui podran suceder X cosas:\\
Llamemos N\_l al ultimo nodo del camino que conecta con el nodo visitado. Y N\_v al nodo ya visitado.

Si la cantidad de paredes rotas para llegar a N\_v (N\_l + 1 si N\_v es pared o N\_l) es mayor o igual a la de  N\_v, este nunca ser\'a actualizado ni encolado
Si la cantidad de paredes rotas para llegar a N\_v (N\_l + 1 si N\_v es pared o N\_l) es menor a la cantidad de paredes rotas de N\_v, entonces, este ser\'a encolado y actualizado. Tomemos N\_v' como el nodo actualizado.
N\_v' tendr\'a una cantidad de paredes rotas mayor o igual que N\_l por lo mencionado anteriormente. Ahora bien, cualquier nodo visitado de un camino que llegue a N\_v' tendrá cantidad de paredes rotas menor o igual a la cantidad de paredes rotas que habia en N\_v. 


