Para demostrar que nuestro algoritmo es válido para solucionar el problema en cuestion, tendremos que probar las siguientes condiciones:\\

\begin{itemize}
\item Si el grafo analizado tiene solución desde un nodo origen a un nodo destino dentro del limite de paredes a romper, el algoritmo siempre termina y si no posee solución tambien finaliza dado que no existen ciclos infinitos.
\item El camino resultante es el m\'as corto dentro de los posibles, es decir, dentro de los que no exceden la cantidad de paredes rompibles.
\end{itemize}

Como nuestro algoritmo puede encolar nodos ya visitados queremos ver que no se producen ciclos infinitos entre nodos con la condicion mencionada.

Dado que el algoritmo no encola nodos de los bordes del grafo, no existirá la posibilidad de generarse ningún ciclo llegando a un borde.

Supongamos que se producen ciclos infinitos entre nodos ya visitados. 

Sea $N_1$ el nodo visitado del que se parte un recorrido, con una cierta cantidad de paredes rotas y $C$ el ciclo de nodos visitados desde $N_1$ hasta otro nodo  $N_k$  que pasa por $N_2$, $N_3$ ... $N_{k-1}$ y que se vuelven a actualizar con menor cantidad de paredes derribadas (dado que si esto no sucede, no hay circuito porque no se actualizan todos los nodos del mencionados).\\
 
La cantidad de paredes rotas del camino desde $N_1$ hasta $N_k$ es creciente porque hasta $N_k$ fueron actualizados todos los nodos del camino.\\

Para poder actualizar $N_1$ pasando por $N_k$, la cantidad de paredes rotas de $N_k$ debería ser menor a la cantidad de paredes rotas en $N_1$. Es decir:

\begin{equation}
paredesRotas(N_1) <
\Bigg \{
\begin{matrix} 
paredesRotas(N_k) + 1 & \mathrm{si\ } N_1 \text{$\text{ es pared}$} \\
paredesRotas(N_k)  & \mathrm{si no\ }
\end{matrix}
\end{equation}

Pero esto no es cierto dado que sabemos que cualquier camino es creciente mayor o igual en cantidad de paredes rotas. 
Por lo tanto la cantidad de paredes rotas en $N_k$ es mayor o igual que $N_1$ y el ciclo no podrá cerrarse dado que el algoritmo no actualiza. Absurdo!! que viene de suponer que se producen ciclos infinitos entre nodos visitados.

Por lo tanto, no habrá ciclos infinitos entre nodos visitados.\\

Si hay solución, no se excede la cantidad de paredes rompibles para llegar al destino, por lo tanto el algoritmo encolará todos los nodos que lleven al destino y en particular aquellos que estén conectados al mismo. Entonces, el destino será actualizado y el algoritmo finalizará su ejecución.
