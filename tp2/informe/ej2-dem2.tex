Como no interesa ninguna arista dentro de cada "habitacion" para hallar la solución, al asignarle el peso nulo se asegura que no afecte en el calculo del peso del árbol generado por Kruscal. Tampoco interesa las aristas seleccionadas por el algoritmo dentro de cada componente, ya que el peso de cada una siempre será cero. Como lo que interesa es elegir solo 1 pared entre todas las que unen 2 habitaciones, entonces al calcular un AGM entre el grafo que las contiene:
\begin{itemize}
\item Solo se eligirá 1, ya que de lo contrario se generaría un ciclo entre ambas componentes.
\item La arista elegida será la de menor peso.
\item El peso del árbol será independiente al arbol generado dentro de cada componente con las aristas sin peso, con lo cual las aristas seleccionadas son irrelevantes
\end{itemize}

En consecuencia de los puntos anteriores, el peso del árbol generado por Kruscal sobre el grafo, representa la suma minima del peso de las aristas necesarias para generar un grafo conexo y ,como las aristas pesadas representan a las paredes del mapa, también representa el esfuerzo minimo para unir todas las habitaciones en el mapa. 


