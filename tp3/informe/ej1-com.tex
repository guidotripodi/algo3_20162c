Considerando el arbol de todas las decisiones posibles para formar un camino, la cantidad de pasos que efectua el algoritmo se ve reducida por las podas efectuadas:

Dado que en un camino no pueden repetirse lugares, la longitud esta acotada por $n+m$, que representaría al camino que pasa 1 vez por cada lugar. La cantidad de caminos de este estilo esta dada por la permutación de $n+m$ elementos en $n+m$ lugares, es decir $O\Big ( (n+m)! \Big)$.\\

La poda de distancia actúa en la medida en que el orden en que se evalúan los caminos. En el caso en que el orden fuese decreciente y se evalúen los caminos con mayor longitud primero, entonces la poda nunca se efectuará.

La poda de cantidad de pociones necesarias elimina todos los casos sin solución en $O(n+1)$ que es lo que se demora en leer la entrada del algoritmo, aportandonos de esta forma el mejor caso para nuestro algoritmo ($\Omega(n+m)$).

La poda que evitan ir a una poke parada de tener todas las pociones posibles actuará de mejor forma en el caso en que se tegan una mayor cantidad de poke paradas que de gimnasios, evitando caminos innecesariamente largos a travez de sucesivas pokeparadas. Asi también el caso en que haya muchos gimnasios y pocas pokeparadas, la poda de acceder a gimnasios solo si se tiene la cantidad necesaria de pociones recortará todos los caminos que recorran innecesariamente gimnasios, perdiendo. 

Se puede observar que la poda que evita ir a una pokeparada de tener la mochila de capacidad $k$ completa nos indica que entre cada viaje entre gimnasios, en el peor de los casos se visitan $\frac{k}{3}$ pokeparadas, lo que nos permite acotar a la complejidad por $O((\frac{k}{3}m + m)!)$ en el caso en que $n\geq\frac{k}{3}m$ $O\Big(min\big\{ (n+m)!, (\frac{k}{3}m + m)! \big\}\Big) \in O\big((n+m)!\big)$.
 