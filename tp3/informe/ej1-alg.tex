\begin{algorithm}[H]
\Fn{EJ1()}{
crear cola desiciones para guardar elecciones\\
eleccion contiene un id, posicion de la misma y distancia desde el punto anterior hasta el como tambien la cantidad de pociones hasta el momento\\
se crean booleanos sePudoDeshacerEleccion con valor verdadero \hfill O(1) \\
se crea un entero minimo con valor -1 \hfill O(1)\\ 
\While{sePudoDeshacerEleccion}{  
	\If{soyElMejor(MaestroPokemon) $\wedge$ tiempo(MaestroPokemon) < minimo}{
	 \hfill  guarda: O(1)\\
			para minimo asignar tiempo(MaestroPokemon) \hfill O(1)\\	
			para caminoRecorrido asignar caminoRecorrido(MaestroPokemon) \hfill O(1)\\	
	}
			eleccion asignar eleccionPosible(MaestroPokemon) \hfill O($\binom{n+m}{1} \times (n + m)$)\\			
			aplicarEleccion(eleccion, MaestroPokemon)\hfill O(n + m)\\
		\Else{
			sePudoDeshacerEleccion asginar deshacerEleccion(MaestroPokemon) \hfill O(n)\\
		}}
		devolver minimo, tamaño(caminoRecorrido) y caminoRecorrido 	\hfill O(n + m)\\
		}
		
		\textbf{\hfill total: $O((n+m)^{2^{n+m}})$}\\
\end{algorithm}

\begin{algorithm}[H]
\Fn{aplicarEleccionEnvio(eleccion, escenario)}{
	encolar en decisiones la eleccion tomada  \hfill O(1) \\
	actualizar estado del sistema en base a la decision tomada \hfill 0(1) \\
	sumar a escenario.tiempo tiempo(eleccion) \hfill O(1) \\
	guardar estado isla A y estado isla B como efectuado\hfill O(n) \\
	marcar escenario.lampara falso \hfill O(1) \\

}
\textbf{\hfill Complejidad total: $O(n)$}\\ 
\end{algorithm}

\begin{algorithm}[H]
\Fn{aplicarRetornoEnvio(eleccion, escenario)}{
	encolar en decisiones la eleccion tomada  \hfill O(1) \\
	actualizar estado del sistema en base a la decision tomada \hfill 0(1) \\
	sumar a escenario.tiempo tiempo(eleccion) \hfill O(1) \\
	guardar estado isla A y estado isla B como efectuado \hfill O(n) \\
	marcar escenario.lampara verdadero \hfill O(1) \\
}
\textbf{\hfill Complejidad total: $O(n)$}\\ 
\end{algorithm}

\begin{algorithm}[H]
\Fn{deshacerUltimaEleccionEnvio(escenario)}{
	\If{hay decisiones para desencolar}{
	desencolar de decisiones una eleccion\hfill O(1) \\
	actualizar estado del sistema en base a la decision desencolada \hfill 0(1) \\
	restar a escenario.tiempo tiempo(eleccion) \hfill O(1) \\
	borrar estado isla A y estado isla B como efectuado \hfill O(n) \\
	marcar escenario.lampara verdadero \hfill O(1) \\	
	devolver verdadero \hfill O(1) \\	
	}\Else{ 
		devolver falso \hfill O(1) \\	
	}
}
\textbf{\hfill Complejidad total: $O(n)$}\\ 
\end{algorithm}


\begin{algorithm}[H]
\Fn{deshacerUltimoRetorno(escenario)}{
	\If{hay decisiones para desencolar}{
	desencolar de decisiones una eleccion\hfill O(1) \\
	actualizar estado del sistema en base a la decision desencolada \hfill 0(1) \\
	restar a escenario.tiempo tiempo(eleccion) \hfill O(1) \\
	borrar estado isla A y estado isla B como efectuado \hfill O(n) \\
	marcar escenario.lampara falso \hfill O(1) \\	
	devolver verdadero \hfill O(1) \\	
	}\Else{ 
		devolver falso \hfill O(1) \\	
	}
}
\textbf{\hfill Complejidad total: $O(n)$}\\ 
\end{algorithm}

\begin{algorithm}[H]
\Fn{eleccionEnvioPosible(escenario)}{
	\While{Hay elecciones disponibles}{
	\hfill Ciclo: O($\binom{n}{2}$)\\
	
	\If{ambas personas estan en isla A $\wedge$ no ocurri\'o elecci\'on $\wedge$ cantidades balanceadas}{
	 \hfill Guarda: O(n) \\	
	devolver elecci\'on \hfill O(1) \\	
	}
	}
}
\textbf{\hfill Complejidad total: $O\binom{n}{2} \times n$}\\ 
\end{algorithm}

\begin{algorithm}[H]
\Fn{eleccionRetornoPosible(escenario)}{
	\While{Hay elecciones disponibles}{
	\hfill Ciclo: O($\binom{n}{2}$)\\
	
	\If{ambas personas estan en isla B $\wedge$ no ocurri\'o elecci\'on $\wedge$ cantidades balanceadas}{
	 \hfill Guarda: O(n) \\	
	devolver elecci\'on \hfill O(1) \\	
	}
	}
}
\textbf{\hfill Complejidad total: $O\binom{n}{2} \times n$}\\ 
\end{algorithm}


\subsubsection{Estructura interna para chequear elecciones y estados ocurridos}

Siendo que la disposici\'on de las personas en las dos islas,pueden representarse por un arreglo en donde cada elemento representa la presencia de una persona en la isla A con un booleano (Esta o no esta) entonces, si guardamos dichas secuencias en una estructura de trie, podemos chequear la ocurrencia de un estado en $O(n)$, al igual que el insertado de un estado nuevo.\\

