Queremos observar que tan bueno es el algoritmo propuesto en la práctica. Para esto realizaremos una serie de tests utilizando las entradas que se utilizaron en el ejericio 3. Además, agregaremos los casos de entrada con solución óptima, ya que en principio, se desconoce la calidad del resultado y el algoritmo no tiene un tiempo de parada prefijado, con lo que estos casos pueden aportar conocimientos nuevos sobre el comportamiento de tabú search.\\
La idea es tratar de encontrar la configuración ideal para que en promedio el algoritmo resuelva la mayoria de los casos eficientemente. Para lograr esto, debemos experimentar variando distintos parametros para cada entrada y obtener las respectivas conclusiones de los resultados obtenidos.\\
Los parámetros de estudio serán:

\begin{enumerate}
\item  \textbf{Cantidad de iteraciones}: Tenemos que observar en promedio, cuantas iteraciones conviene tomar para obtener un buen resutado.
\item \textbf{Tenor tabú}: Tenemos que observar que sucede al aumentar el tenor tabú, y hasta cuanto es conveniente hacerlo para obtener un buen resultado.
\item \textbf{Atributos tabú}: Analizar que sucede al tomar como atributos tabú las aristas que cambiaron o las aristas nuevas
\item \textbf{Función de aspiración}: Elegir la solución más tabú o la menos tabú.
\end{enumerate}

Se tomarán 30 mediciones por cada tipo de test. En el caso de 1) y 2), por cada aumento realizado. Se tomará una media alfa podada de las mismas con $\alpha$ = 0.5 de manera de podar un 25\% de los datos a cada lado. De esta forma se reducen los outliers en las muestras consideradas. 
Además se tomará la varianza muestral usando la media calculada y las mediciones que queden luego de aplicar la media.\\

REALIZAR LOS TESTS PARA CADA ENTRADA. OBTENER CONCLUSIONES. REALIZAR NUEVOS TESTS. OBTENER CONFIGURACIONES. OBSERVAR QUE TAN BUENA ES LA SOLUCION ES PARTE DE LAS CONCLUSIONES!!!.
