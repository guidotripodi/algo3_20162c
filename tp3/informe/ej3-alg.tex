\begin{algorithm}[H] %or another one check
 \Fn{swap()}{
 %     \SetAlgoLined
   $S_o$ es un Recorrido solución que brinda el algoritmo goloso \\
  	n es cantidad de nodos de $S_o$ \\
  	Recorrido $S_{actual}$ $\leftarrow$ $S_o$ \hfill O(n)\\
  	entero costoAnterior $\leftarrow$ calcularCosto($S_o$) \hfill O(1)\\
  	Recorrido $S_{final}$ $\leftarrow$ $S_o$ \hfill O(n)\\
	hayMejora $\leftarrow$ true \hfill O(1)\\
	\While{hayMejora}{ 
	\hfill ciclo: O(n - longitud solucion optima)\\
		entero costoActual $\leftarrow$ -1 \\
		\For{i de 1 a n}{
		\hfill ciclo: O(n)\\	
			\For{j de i+1 a n}{ 
			\hfill ciclo: O(n)\\	
				 $intercambiar$ $posiciones$ i con j en $S_{actual}$ \hfill O(1)\\
				 optimizarS($S_{actual}$) \hfill O(n)\\
				 costoActual $\leftarrow$ calcularCosto($S_{actual}$) \hfill O(n)\\
				 \If{costoActual $<>$ -1 $\wedge$ costoActual < costoAnterior}{
					costoAnterior $\leftarrow$ costoActual \hfill O(1)\\
					$S_{final}$ $\leftarrow$ $S_{actual}$ \hfill O(n)\\	 
				}
				 $intercambiar$ $posiciones$ i con j en $S_{actual}$ \hfill O(1)\\	 		
			}
		}
		\If{costoActual = -1 $\vee$ costoActual $\geq$ costoAnterior}{
			hayMejora $\leftarrow$ false \hfill O(1)\\	 
		}
	}
	devolver $S_{final}$ \\	
	
	\hfill complejidad total: O($n^3$)\\			
}
\end{algorithm}

\begin{algorithm}[H] %or another one check
 \Fn{2opt()}{
 %     \SetAlgoLined
   Recorrido $S_o$ es la solución que brinda el algoritmo goloso \\
  	n es cantidad de nodos de $S_o$ \\
  	Recorrido $S_{actual}$ $\leftarrow$ $S_o$ \hfill O(n)\\
  	entero costoAnterior $\leftarrow$ calcularCosto($S_o$) \hfill O(1)\\
  	Recorrido $S_{final}$ = $S_o$ \hfill O(n)\\
	\While{hayMejora}{ 
	\hfill ciclo: O(n - longitud solucion optima)\\
		entero costoActual $\leftarrow$ -1 \\
		\For{i de 1 a n}{
		\hfill ciclo: O(n)\\
			\For{j de i+1 a n}{
			\hfill ciclo: O(n)\\
				 $invertir$ $rango$ de i a j en $S_{actual}$ \hfill O(n)\\
				 optimizarS($S_{actual}$) \hfill O(n)\\
				 costoActual $\leftarrow$ calcularCosto($S_{actual}$) \hfill O(n)\\
				 \If{costoActual $<>$ -1 $\wedge$ costoActual < costoAnterior}{
					costoAnterior $\leftarrow$ costoActual \hfill O(1)\\
					$S_{final}$ $\leftarrow$ $S_{actual}$ \hfill O(n)\\
				}
				 $invertir$ $rango$ de i a j en $S_{actual}$ \hfill O(n)\\
			}	
		}
		\If{costoActual = -1 $\vee$ costoActual $\geq$ costoAnterior}{
			hayMejora $\leftarrow$ false \hfill O(1)\\	 
		}
	}
	
	devolver $S_{final}$ \\	
	
	\hfill complejidad total: O($n^3$)\\	
}
\end{algorithm}

\begin{algorithm}[H] %or another one check
 \Fn{3opt()}{
 %     \SetAlgoLined
   Recorrido $S_o$ es la solución que brinda el algoritmo goloso \\
  	n es cantidad de nodos de $S_o$ \\
  	Recorrido $S_{actual}$ $\leftarrow$ $S_o$ \hfill O(n)\\
  	entero costoAnterior $\leftarrow$ calcularCosto($S_o$) \hfill O(1)\\
  	Recorrido $S_{final}$ = $S_o$ \hfill O(n)\\
  	hayMejora $\leftarrow$ true	
	\While{hayMejora}{	
	\hfill ciclo: O(n - longitud solucion optima)\\
		entero costoActual $\leftarrow$ -1 \\
		\For{i de 1 a n-3}{
		\hfill ciclo: O(n)\\
			\For{j de i+1 a n-2}{
			\hfill ciclo: O(n)\\
					\For{k de j+2 a n}{
					\hfill ciclo: O(n)\\
					
				 caso 1:\\
				 $invertir$ $rango$ de i a j en $S_{actual}$ \hfill O(n)\\
				 $invertir$ $rango$ de j+1 a k en $S_{actual}$ \hfill O(n)\\
				 optimizarS($S_{actual}$) \hfill O(n)\\
				 entero costoActual $\leftarrow$ calcularCosto($S_{actual}$) \hfill O(n)\\
				 \If{costoActual $<>$ -1 $\wedge$ costoActual < costoAnterior}{
					costoAnterior $\leftarrow$ costoActual \hfill O(1)\\
					$S_{final}$ = $S_{actual}$ \hfill O(n)\\
				 }
				 $invertir$ $rango$ de j+1 a k en $S_{actual}$	\hfill O(n)\\	 
				 $invertir$ $rango$ de i a j en $S_{actual}$ \hfill O(n)\\
				 
				 
				 caso 2: \\
				$intercambiar$ $rango$ de i a j con el de j+1 a k en $S_{actual}$ \hfill O(n)\\
				 optimizarS($S_{actual}$) \hfill O(n)\\
				 entero costoActual $\leftarrow$ calcularCosto($S_{actual}$) \hfill O(n)\\
				 \If{costoActual $<>$ -1 $\wedge$ costoActual < costoAnterior}{
					costoAnterior $\leftarrow$ costoActual \hfill O(1)\\
					$S_{final}$ = $S_{actual}$ \hfill O(n)\\
				 }
				 $intercambiar$ $rango$ de i a j con el de j+1 a k en $S_{actual}$ \hfill O(n)\\
				 
				 
				 caso 3:\\
				 es igual al caso 2 pero además invirtiendo el rango i a j \hfill O(4*n)\\
				 
				 
				 caso 4: \\
				 es igual al caso 2 pero además invirtiendo el rango j+1 a k \hfill O(4*n)\\
				}
			}	
		}	
	}
	
		\If{costoActual = -1 $\vee$ costoActual $\geq$ costoAnterior}{
			hayMejora $\leftarrow$ false \hfill O(1)\\	 
		}

	devolver $S_{final}$ \\

	\hfill complejidad total: O($n^4$)\\
}

\end{algorithm}

\begin{algorithm}[H]

\Fn{optimizarS($S_o$)}{
	\While{back($S_o$).tipo = $pokeparada$}{
		pop\_back($S_o$)\\
	}
}

\end{algorithm}

\begin{algorithm}[H]

\Fn{calcularCosto(Recorrido camino)}{
	entero costo = 0 \hfill O(1) \\
	entero capacidadParcial = 0 \hfill O(1) \\
	
	\For{i desde 2 hasta |camino|}{
	\hfill ciclo: O(n) \\
		\If{pasoPosible(camino[i], capacidadParcial)}{
			\hfill guarda: O(1) \\
			$<entero, entero>$ pOrigen \\
			$<entero, entero>$ pDestino \\
			
			entero origen $\leftarrow$ camino[i-1]  \hfill O(1) \\
			entero destino $\leftarrow$ camino[i] \hfill O(1) \\
			
			bool destinoEsPP $\leftarrow$ false \hfill O(1) \\
			
			\If{origen $<$ cantGyms}{
				pOrigen $\leftarrow$ gimnasiosArr[origen].coord \hfill O(1) \\
			}\Else {
				pOrigen $\leftarrow$ pokeParadasArr[origen-cantGyms] \hfill O(1) \\
			}
			
			\If{destino $<$ cantGyms}{
				pDestino $\leftarrow$ gimnasiosArr[destino].coord \hfill O(1) \\
			}\Else {
				pDestino $\leftarrow$ pokeParadasArr[destino-cantGyms] \hfill O(1) \\
				destinoEsPP $\leftarrow$ true \hfill O(1) \\
			}
		
		    costo $\leftarrow$ costo + distanciaEuclidea(pOrigen, pDestino) \hfill O(1) \\
			
			\If{destinoEsPP}{
				capacidadParcial += 3 \hfill O(1) \\
				\If{capacidadParcial $>$ capMochila}{
					capacidadParcial $\leftarrow$ capMochila \hfill O(1) \\
				}
			}\Else {
				capacidadParcial $\leftarrow$ capacidadParcial - gimnasiosArr[destino].poder \hfill O(1) \\
			}	
					
		}\Else{
			devolver -1 \\
		}
	}
	
	devolver costo \\
	
	\hfill complejidad total: O(n) \\
}

\end{algorithm}
	
\begin{algorithm}[H]

\Fn{pasoPosible(entero destino, entero capacidadParcial}{

	entero poderGym $\leftarrow$ 0 \hfill O(1) \\

	\If {destino $<$ cantGyms}{
		poderGym $\leftarrow$ gimnasiosArr[destino].poder \hfill O(1) \\
	}
	
	\If {poderGym = 0 $\vee$ capacidadParcial $>$ poderGym}{
		devolver true \hfill O(1) \\
	}
	
	devolver false \\
	
	\hfill complejidad total: O(1) \\
}

\end{algorithm}

\begin{itemize}
\item Justificación de la cota del ciclo principal: El algoritmo comienza con una solucion de $n$ nodos. Cada mejora en el peor de los casos se realiza en una unidad, ya que se trabaja con distancias enteras. Como se busca llegar al óptimo global, en el peor de los casos habrá O($n$-longitud de la solución óptima) iteraciones para el ciclo principal.
\item Recorido = Lista de enteros
\item $n$ = |$S_o$|
\item Todos los enteros en un Recorrido estan entre 1 y $N$ con $N$ la cantidad total de nodos (pokeparadas y gimnasios)
\item Los primeros $M$ indices, con $M < N$, es la cantidad de gimnasios (cantGyms) y los restantes $N-M$ indices,  pokeparadas. Luego hay un arreglo de gimnasios para los primeros M indices y uno de pokeparadas para los últimos $N-M$ indices.
\item Gimnasio = < <entero x, entero y>, entero poder>
\item PokeParada = <entero x, entero y>
\item gimnasiosArr es un arreglo de Gimnasio
\item pokeParadasArr es un arreglo de PokeParada
\end{itemize}

Podemos ver que todos los algoritmos iteran sobre la solución $S_o$, que en el peor caso puede contener todos los nodos del mapa, osea, $n=N$.

Las operaciones $invertir$ $rango$ o $intercambiar$ $rango$ en el peor caso serán realizadas sobre los $n$ nodos de $S_o$.

La operación costo es O(n) ya que requiere recorrer $S_o$ hasta la última posición observando si un movimiento es válido. Recordemos que la validez de un movimiento se observa cuando se avanza hacia un gimnasio. Este movimiento será válido si y solo si se puede vencer al gimnasio. Esto último es un chequeo que puede realizarse en tiempo constante.  

Luego, realizar busquedas locales con las vecindades planteadas es, en el peor caso, de complejidad polinómica. 


