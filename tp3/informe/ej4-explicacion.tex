Como su nombre lo indica, búsqueda local analiza una vecindad  local a la solución inicial $S_o$. Por lo que generalmente, la mejora obtenida puede no ser global, si no, la mejor solución dentro de la vecindad analizada. 

Para salir de un óptimo local, existen meta heurisiticas que pueden o no proveer una mejor solución observando otras vecindades, y en algunos casos acercarse lo suficiente u obtener el óptimo global. Una de ellas es la elegida para este informe, denominada, $tab\'u$ $search$.\\

La idea de esta meta heuristica es ir moviendose por las vecindades subyacentes a una vecindad analizada. Es decir, las vecindades de las soluciones que conforman una vecindad.
Pero no todas ellas, si no la vecindad de una solución elegida que cumpla con ciertos atributos, o mejor dicho, que no posea ciertas caracteristicas. 
Esto es así, dado que se tiene que buscar una manera de descartar soluciones que no se consideren adecuadas para ser analizadas, si no, caeriamos en el problema de backtracking, donde se consideran todas las posibilidades, lo cual, puede ser impracticable.\\
Los atributos elegidos 

 
 